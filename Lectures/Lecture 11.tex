\vspace*{1em}

\begin{mdframed}
\begin{center}
{\Large Integral of $\gamma:[a,b] \to \cc$}
\end{center}
\end{mdframed}

\begin{definition}[Definite Integral of $\gamma$]
Consider a function $\gamma:[a,b] \to \cc$ with \[\gamma(t) = u(t) + iv(t),\] where $u,\,v: [a,b] \to \rr$. The \cdef{definite\ integral\ of} {\color{blue}$\gamma$} is defined as
\[\int_a^b\gamma(t)\ dt \coloneqq \int_a^bu(t)\ dt + i\int_a^bv(t)\ dt\]
provided the integrals of $u$ and $v$ exist.\\
\\
Improper integrals can be defined in a similar manner.
\end{definition}

\vspace*{1em}

\begin{example}
We illustrate this definition by integrating $\gamma(t) = e^{it}$ on $[0,\pi]$.
\begin{align*}
\int_0^\pi e^{it}\ dt &= \int_0^\pi\cos t\ dt + \int_0^\pi\sin t\ dt\\[1em]
&= \Big[\sin t\,\Big]_0^\pi + i\Big[-\cos t\,\Big]_0^\pi\\[1em]
&= (\sin \pi - \sin 0) + i(-\cos\pi + \cos 0) = 2i
\end{align*}
\end{example}

\vspace*{1em}

\begin{definition}[Piecewise Continuity]
A function $u:[a,b] \to \rr$ is \cdef{piecewise\ continuous\ on} {\color{blue}$[a,b]$} if it is continuous on $[a,b]$ except at a finite number of points, where despite its discontinuity on those points, both one sided limits exist.\\
\\
We call $\gamma(t) = u(t) + iv(t)$ \emph{piecewise continuous} if both $u$ and $v$ are. 
\end{definition}

\vspace*{1em}

\begin{remark}
The existence of the integrals 
\[\int_a^bu(t)\ dt \quad \text{and} \quad \int_a^b v(t)\ dt\]
is guaranteed when $\gamma$ is piecewise continuous.
\end{remark}

\vspace*{1em}

\begin{proposition}[Properties of the Integral of $\gamma$]\label{paraint}
Suppose $\gamma$ and $\gamma_1$ are piecewise continuous on $[a,b]$, then
\begin{itemize}[itemsep=0.5em]
\item[(1)] $\displaystyle \int_a^bz_0 \gamma(t)\ dt = z_0\int_a^b\gamma(t)\ dt$, for any $z_0 \in \cc$.
\item[(2)] $\displaystyle \int_a^b \gamma(t) + \gamma_1(t)\ dt = \int_a^b\gamma(t)\ dt + \int_a^b\gamma_1(t)\ dt$.
\item[(3)] $\displaystyle \int_a^b \gamma(t)\ dt = \int_a^c\gamma(t)\ dt + \int_c^b\gamma(t)\ dt$, for any $c \in [a,b]$.
\item[(4)] $\displaystyle \int_b^a \gamma(t)\ dt = -\int_a^b\gamma(t)\ dt$.
\end{itemize}
\end{proposition}
\begin{proof}
These properties follow from the properties of regular real integrals applied to the real and imaginary part of $\gamma$ and $\gamma_1$.
\end{proof}

\vspace*{1em}

\begin{proposition}[Extension of Fundamental Theorem of Calculus]\label{pathftc}
Suppose that $\gamma(t) = u(t) + iv(t)$ is continuous on $[a,b]$ and $\Gamma(t) = U(t) + iV(t)$ is differentiable such that $\Gamma'(t) = \gamma(t)$ on $[a,b]$. Then
\[\int_a^b\gamma(t)\ dt = \Gamma(b) - \Gamma(a)\]
\end{proposition}
\begin{proof}
By assumption $\Gamma' = \gamma$, therefore $U'(t) = u(t)$ and $V'(t) = v(t)$, therefore
\begin{align*}
\int_a^b\gamma(t)\ dt &= \int_a^bu(t)\ dt + i\int_a^bv(t)\ dt\\[0.5em]
&= U(b) - U(a) + i(V(b) - V(a)),\ \text{by the Fundamental Theorem of Calculus}\\[0.5em]
&= U(b) + iV(b) - (U(a) + iV(a))\\[0.5em]
&= \Gamma(b) - \Gamma(a)\\[-2.5em]
\end{align*}
\end{proof}

\vspace*{1em}

\begin{example}
We use this proposition to integrate $e^{it}$ on $[0,\pi]$. For this, we first note that
\[\left(\frac{e^{it}}{i}\right)' = \frac{1}{i}\left(e^{it}\right)' = \frac{i}{i}\,e^{it} = e^{it}.\]
Therefore, 
\begin{align*}
\int_0^\pi e^{it}\ dt = \Bigg[\frac{e^{it}}{i}\Bigg]_0^\pi &= \Big[-ie^{it}\Big]_0^\pi\\[0.5em]
&= -ie^{i\pi} + ie^{i\cdot 0}\\[0.5em]
&= i + i = 2i
\end{align*}
\end{example}

\vspace*{2em}

\begin{mdframed}
\begin{center}
{\Large Contours}
\end{center}
\end{mdframed}

So far, we have only defined the integral of a complex-valued function in a single real variable over an interval. Integrals of complex-valued functions in a single complex variable are defined over suitable curves in the complex plane called \emph{contours}.

\vspace*{1em}

\begin{definition}[Arcs]\hfill
\begin{itemize}
\item[(1)] An \cdef{arc}, or \cdef{curve}, is a collection of points 
\[C = \setp{z(t)}{t \in [a,b]},\]
where $z(t) = x(t) + iy(t)$ and $x,\,y: [a,b] \to \rr$ are continuous functions. The function $z(t)$ is called a \cdef{parametrization\ of} {\color{blue}$C$}.
\item[(2)] An arc (or curve) $C$ is called \cdef{simple} or a \cdef{Jordan\ arc} if it does not cross itself, which is equivalent to saying the function $z(t)$ is injective; that is, if $z(t_1) = z(t_2)$ then $t_1 = t_2$. 
\item[(3)] If $C$ is simple except for the fact that $z(a) = z(b)$, then $C$ is called a \cdef{simple\ closed\ curve} or a \cdef{Jordan\ curve}.
\item[(4)] A simple closed curve is \cdef{positively\ oriented} if it is transversed counter-clockwise as $t$ increases from $a$ to $b$. It is called \cdef{negatively\ oriented} if it is transversed clockwise.
\end{itemize}
\end{definition}

\vspace*{1em}

\begin{center}
\begin{minipage}{0.33\textwidth}
\[\begin{tikzpicture}[scale=0.5]
    \draw[<->,thick] (-2,0)--(5,0);
	\draw[<->,thick] (0,-2)--(0,5);
    \begin{scope}
        \node(A) at (-2,-2) {};
        \node(B) at (4.5,4.5){};
        \draw[use Hobby shortcut,clockwise arrows,thick]
	(4.5,4.5) .. (2,2) .. (2.5,0.5) .. (2,2) .. (-0.5,0.5) .. (-2,-2);
    \end{scope}
    \draw [fill=black] (A) circle (2pt);
    \draw [fill=black] (B) circle (2pt);
\end{tikzpicture}\]
\begin{center}
\emph{a not simple arc}
\end{center}
\end{minipage}
\begin{minipage}{0.33\textwidth}
\[\begin{tikzpicture}[scale=0.5]
    \draw[<->,thick] (-2,0)--(5,0);
	\draw[<->,thick] (0,-2)--(0,5);
    \begin{scope}
        \node(A) at (-2,-2) {};
        \node(B) at (4.5,4.5){};
        \draw[use Hobby shortcut,clockwise arrows,thick]
	(4.5,4.5) .. (3,2) .. (0,1) .. (-2,-2);
    \end{scope}
    \draw [fill=black] (A) circle (2pt);
    \draw [fill=black] (B) circle (2pt);
\end{tikzpicture}\]
\begin{center}
\emph{a simple arc}
\end{center}
\end{minipage}

\vspace*{3em}

\begin{minipage}{0.325\textwidth}
\[\begin{tikzpicture}[scale=0.5]
    \draw[<->,thick] (-2,0)--(5,0);
	\draw[<->,thick] (0,-2)--(0,5);
    \begin{scope}
        \draw[use Hobby shortcut,closed=true,wise arrows,thick]
	(4.5,4.5) .. (4,2) .. (3,-2) .. (0,-1) .. (-1.5,-1.5) .. (-1,0) .. (-2,3) .. (-1,4.5) .. (2,4) .. (4.5,4.5);
    \end{scope}
\end{tikzpicture}\]
\begin{center}
\emph{a simple closed curve\\ with positive orientation}
\end{center}
\end{minipage}
\begin{minipage}{0.325\textwidth}
\[\begin{tikzpicture}[scale=0.5]
    \draw[<->,thick] (-2,0)--(5,0);
	\draw[<->,thick] (0,-2)--(0,5);
    \begin{scope}
        \draw[use Hobby shortcut,closed=true,wise arrows,thick]
	(0,4) .. (-2,2) .. (0,0) .. (0.5,-0.5) .. (0.5,-1) .. (3,-0.75) .. (4,-1) .. (3.5,0) .. (4,1) .. (2.5,2.5) .. (1,2) .. (0,4);
    \end{scope}
\end{tikzpicture}\]
\begin{center}
\emph{a simple closed curve\\ with negative orientation}
\end{center}
\end{minipage}
\begin{minipage}{0.325\textwidth}
\[\begin{tikzpicture}[scale=0.5]
    \draw[<->,thick] (-3.5,0)--(3.5,0);
	\draw[<->,thick] (0,-3.75)--(0,3.75);
    \begin{scope}
        \draw[use Hobby shortcut,closed=true,wiser arrows,thick]
	(0,3) .. (1.5,1.5) .. (0,0) .. (-1.5,-1.5) .. (0,-3) .. (1.5,-1.5) .. (0,0) .. (-1.5,1.5) .. (0,3);
    \end{scope}
\end{tikzpicture}\]
\begin{center}
\emph{a not simple closed\\ non-orientable curve}
\end{center}
\end{minipage}
\end{center}

\vspace*{1em}

\begin{example}
The most frequently encountered arcs and curves are line segments and circles. 
\begin{itemize}
\item[(1)] The circle of radius $R$ centered at $z_0$ with positive orientation has as a parametrisation
\[z(t) = z_0 + Re^{it},\quad t \in [0,2\pi]\]
\[\begin{tikzpicture}[scale=0.65]
    \draw[<->,thick] (-1,0)--(5,0);
	\draw[<->,thick] (0,-1)--(0,5);
    \node (a) at (2,4.732) {};
    \node (b) at (3,3) {};
    \node (c) at (5,3) {};
    \draw pic["{\footnotesize$t$}", ->,>=stealth,thick,draw, angle eccentricity=1.5, angle radius=0.5cm,teal] {angle=c--b--a};
	\draw [teal,dashed,thick] (3,3) -- (5.5,3);
	\draw [teal,thick] (3,3) -- (1.6,5.071);
	\draw [thick] (3,3) -- (0.55,2.503)node [midway,above,sloped] {$R$};
    \node[label=below:$z_0$](A) at (3,3) {};
	\draw [fill=black] (A) circle (1.5pt);
    \begin{scope}
        \draw[use Hobby shortcut,closed=true,wise arrows,thick]
	(3,5.5) .. (5.5,3) .. (3,0.5) .. (0.5,3);
    \end{scope}
\end{tikzpicture}\]

\item[(2)] The circle of radius $R$ centered at $z_0$ with negative orientation has as a parametrisation
\[z(t) = z_0 + Re^{-it},\quad t \in [0,2\pi]\]\\[-1em]
\[\begin{tikzpicture}[scale=0.65]
    \draw[<->,thick] (-1,0)--(5,0);
	\draw[<->,thick] (0,-1)--(0,5);
    \begin{scope}
        \draw[use Hobby shortcut,closed=true,wise arrows,thick]
	(3,5.5) .. (0.5,3) .. (3,0.5) .. (5.5,3);
    \end{scope}
\end{tikzpicture}\]

\item[(3)] The line segment from $z_0$ to $z_1$ in $\cc$ has as a parametrisation
\[z(t) = z_0 + (z_1 - z_0)t = (1-t)z_0 + tz_1,\quad t \in [0,1]\]\\[-1em]
\[\begin{tikzpicture}
    \draw[<->,thick] (-1,1.5)--(4,1.5);
	\draw[<->,thick] (0,1)--(0,4);
	\begin{scope}
        \draw[use Hobby shortcut,clockwise arrows,thick]
	(3.5,3.5) .. (-1,3);
    \end{scope}    
    \node[label=below:$z_0$](A) at (-1,3) {};
    \node[label=below:$z_1$](B) at (3.5,3.5){};
	\draw [fill=black] (A) circle (2pt);
    \draw [fill=black] (B) circle (2pt);
\end{tikzpicture}\]
\end{itemize}
\end{example}

\vspace*{1em}

\begin{definition}[Reparametrisation of an arc]
Suppose an arc $C$ is parametrised by $z:[a,b] \to \cc$. A map
\[w:[c,d] \to \cc\]
is called an \cdef{orientation\text{-}preserving\ reparametrisation} of $C$ if there exists a surjective function
\[\phi:[c,d] \to [a,b]\]
with continuous derivative such that $\phi(c) = a$ (preserves initial point), $\phi(d) = b$ (preserves final point), $\phi'(s) > 0$ and $w(s) = z(\phi(s))$ ($w$ and $z$ trace out the same arc $C$).
\end{definition}

%\vspace*{1em}

\begin{example}
Note that $z(t) = e^{it}$ for $t \in [0,2\pi]$ is a parametrisation of the unit circle. Now, consider
\[w:[0,\pi] \to \cc,\ s \mapsto e^{2is},\]
this is, in fact, an orientation-preserving reparametrisation of the unit circle. To conclude this, we produce the following surjective map
\[\phi:[0,\pi] \to [0,2\pi],\ s \mapsto 2s,\]
we note that $\phi(0) = 0$ and $\phi(\pi) = 2\pi$, furthermore $\phi'(s) = 2 > 0$ which is clearly continuous. Lastly, $z(\phi(s)) = z(2s) = e^{2is} = w(s)$.
\end{example}

\vspace*{1em}

\begin{remark}
Suppose an arc $C$ is parametrised by $z:[a,b] \to \cc$, a map $w:[c,d] \to \cc$ 
is called an \cdef{orientation\text{-}reversing\ reparametrisation} of $C$ if there exists a surjective function
\[\psi:[c,d] \to [a,b]\]
with continuous derivative such that $\psi(c) = b$ and $\psi(d) = b$ (swaps initial and final points), $\psi'(s) < 0$ and $w(s) = z(\psi(s))$ ($w$ and $z$ trace out the same arc $C$).\\
\\
Consider the unit circle, which has parametrisation $z(t) = e^{it},\ t \in [0,2\pi]$. Then $w(t) = e^{-it}$ for $0 \leq t \leq 2\pi$ is an orientation-reversing parametrisation. To see this, we consider the surjective function
\[\psi:[0,2\pi] \to [0,2\pi],\ s \mapsto 2\pi - s;\]
we note that $\psi(0) = 2\pi$ and $\psi(2\pi) = 0$, furthermore $\psi'(s) = -1 < 0$ and \[z(\psi(s)) = z(2\pi - s) = e^{2\pi i - is} = e^{-is} = w(s),\] since $e^{2\pi i} = 1$.
\end{remark}

\vspace*{1em}

\begin{definition}[Arc length and Smooth arcs]\hfill
\begin{itemize}
\item[(1)] If $C$ is parametrised by $z(t) = x(t) + iy(t)$ and $x'(t),\,y'(t)$ exist and are continuous on $[a,b]$, then $C$ is called a \cdef{differentiable\ arc}.
\item[(2)] The \cdef{arc\ length} of such a differentiable arc $C$ is
\[L(C) = \int_a^b \abs{z'(t)}\,dt = \int_a^b\sqrt{x'(t)^2 + y'(t)^2}\,dt\]
\item[(3)] A differentiable curve parametrised by $z(t)$ is called \cdef{smooth} if $z'(t) \neq 0$ on $[a,b]$.
\end{itemize}
\end{definition}

\vspace*{2em}

\subsection{Problems}
\vspace{0.1in}
To be added
%\begin{problem}\label{prob 11.1}
%
%\end{problem}