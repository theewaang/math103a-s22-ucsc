\vspace*{1em}

\begin{mdframed}
\begin{center}
{\Large Cauchy-Riemann Equations}
\end{center}
\end{mdframed}

%When considering a real-valued function $f : \rr^2 \to \rr$ of two variables, there is no notion of the derivative of a function. For such a function, we instead only have partial derivatives $f_x(x_0,y_0)$ and $f_y(x_0,y_0)$ (and also directional derivatives) which depend on the way in which we approach a point $(x_0, y_0) \in \rr^2$. For a complex-valued function $f(z)$, we now have a new concept of the derivative $f'(z_0)$, which by definition cannot depend on the way in which we approach a point $z_0 \in \cc$. It is then natural expect a relationship between the complex derivative $f'(z_0)$ and the partial derivatives of $\Re f$ and $\Im f$, which are real-values functions. 
%
%\vspace*{1em}
%
\begin{theorem}[Cauchy-Riemann Equations]\label{crequations}
Suppose that 
\[f(z) = f(x + iy) = u(x,y) + i\,v(x,y)\]
is differentiable at $z_0 = x_0 + iy_0$. Then
\begin{itemize}
\item[(a)] the first order partial derivatives of $u$ and $v$ exist at $(x_0,y_0)$ and satisfy the \cdef{Cauchy\text{-}Riemann\ Equations}
\begin{align*}\label{creqex}
u_x(x_0,y_0) &= v_y(x_0,y_0)\\[-0.5em]
\tag{CR}\\[-0.5em]
u_y(x_0,y_0) &= -v_x(x_0,y_0)
\end{align*}
\item[(b)] $f'(z_0) = u_x(x_0,y_0) + i\,v_x(x_0,y_0) = v_y(x_0,y_0) - i\,u_y(x_0,y_0)$.
\end{itemize}
\end{theorem}
\begin{proof}
Since $f$ is differentiable at $z_0$, we have, where we let $h = s + it$
\begin{align*}
f'(z_0) &= \lim_{h\to 0}\frac{f(z_0 + h) - f(z_0)}{h}\\[0.5em]
&= \lim_{h\to 0}\frac{f((x_0 + s) + i(y_0 + t)) - f(x_0 + iy_0)}{h}\\[0.5em]
&= \lim_{h\to 0}\frac{u(x_0 + s,y_0 + t) - u(x_0,y_0)}{h} + i\cdot\lim_{h\to 0}\frac{v(x_0 + s,y_0 + t) - v(x_0,y_0)}{h}
\end{align*}
As we know by now, we must get the same result if we restrict $h$ to be on the real axis and if we restrict it to be on the imaginary axis. In the former case, $t = 0$, giving us
\begin{align*}
f'(z_0) &= \lim_{s\to 0}\frac{u(x_0 + s,y_0) - u(x_0,y_0)}{s} + i\cdot\lim_{s\to 0}\frac{v(x_0 + s,y_0) - v(x_0,y_0)}{s}\\[0.5em]
&= u_x(x_0,y_0) + i\,v_x(x_0,y_0)
\end{align*}
In the latter case, $s = 0$, giving us
\begin{align*}
f'(z_0) &= \lim_{t\to 0}\frac{u(x_0,y_0 + t) - u(x_0,y_0)}{it} + i\cdot\lim_{t\to 0}\frac{v(x_0,y_0 + t) - v(x_0,y_0)}{it}\\[0.5em]
&= \frac{1}{i}\cdot\lim_{t\to 0}\frac{u(x_0,y_0 + t) - u(x_0,y_0)}{t} + \lim_{t\to 0}\frac{v(x_0,y_0 + t) - v(x_0,y_0)}{t}\\[0.5em]
&= -i\,u_y(x_0,y_0) + v_y(x_0,y_0)
\end{align*}
Therefore
\[u_x(x_0,y_0) + i\,v_x(x_0,y_0) = f'(z_0) = v_y(x_0,y_0)-i\,u_y(x_0,y_0),\]
and hence $u_x(x_0,y_0) = v_y(x_0,y_0)$ and $u_y(x_0,y_0) = -v_x(x_0,y_0)$.
\end{proof}

%\vspace*{1.5em}

The Cauchy-Riemann equations (\ref{creqex}) are a \emph{necessary} condition for $f'$ to exist. We can use them to locate possible points where the derivative does not exist but not necessarily conclude where and if the derivative exists.
\begin{example}\label{necessarycr}\hfill
\begin{itemize}
\item[(1)] Consider $f(z) = \abs{z}^2 = x^2 + y^2$, so $u(x,y) = x^2 + y^2$ and $v(x,y) = 0$. The partial derivatives at $(x,y)$ are
\begin{align*}
u_x &= 2x & v_x &= 0\\[0.5em]
u_y &= 2y & v_y &= 0
\end{align*}
Therefore, the Cauchy-Riemann equations (\ref{creqex}) are only satisfied at $(x,y) = (0,0)$. Hence $f$ is not differentiable at any $z \neq 0$. Again, note that this does not say anything about the existence of $f'(0)$.
\item[(2)] Consider $f(z) = \overline{z} = x - iy$, so $u(x,y) = x$ and $v(x,y) = -y$. The partial derivatives at $(x,y)$ are
\begin{align*}
u_x &= 1 & v_x &= 0\\[0.5em]
u_y &= 0 & v_y &= -1
\end{align*}
Note that $u_x \neq v_y$ for all $(x,y)$ and therefore the Cauchy-Riemann equations (\ref{creqex}) are satisfied for no $(x,y)$. Hence $f$ is nowhere complex-differentiable.
\item[(3)] (in-class) Consider $f(z) = (z + i\overline{z})^2$, let's simplify $f$ to identify its real and imaginary parts $u(x,y)$ and $v(x,y)$.
\begin{align*}
f(z) = f(x + iy) &= ((x + iy) + i(x - iy))^2\\[0.5em]
&= ((x + iy) + (y + ix))^2\\[0.5em]
&= ((x + y) + i(x + y))^2\\[0.5em]
&= (x + y)^2(1 + i)^2\\[0.5em]
&= (x + y)^2(1^2 + i^2 + 2i)\\[0.5em]
&= 2i(x + y)^2
\end{align*}
Therefore $u(x,y) = 0$ and $v(x,y) = 2(x + y)^2$. The partial derivatives at $(x,y)$ are
\begin{align*}
u_x &= 0 & v_x &= 4(x + y)\\[0.5em]
u_y &= 0 & v_y &= 4(x + y)
\end{align*}
Therefore, the Cauchy-Riemann equations (\ref{creqex}) are satisfied if and only if $4(x + y) = 0$, if and only if $y = -x$. Hence $f$ is not differentiable any $z \in \cc$ such that $\Im z \neq -\Re z$.
\end{itemize}
\end{example}

\vspace*{1em}

As commented, the Cauchy-Riemann equations (\ref{creqex}) are not a \emph{sufficient} condition for the existence of the derivative as the example below shows. Problem \ref{prob 7.1} gives another example.
\begin{example}\label{onlynecescr}
Consider
\[f(z) = \begin{cases}\dfrac{\overline{z}^2}{z} = \dfrac{\overline{z}^3}{\abs{z}^2} & z \neq 0\\[1em] 0 & z = 0 \end{cases}\]
Then,
\[u(x,y) = \begin{cases}\dfrac{x^3 - 3xy^2}{x^2 + y^2} & (x,y) \neq (0,0)\\[1em] 0 & (x,y) = (0,0) \end{cases} \qquad \text{and} \qquad v(x,y) = \begin{cases}\dfrac{y^3 - 3x^2y}{x^2 + y^2} & (x,y) \neq (0,0)\\[1em] 0 & (x,y) = (0,0) \end{cases}\]
We show that $u$ and $v$ satisfy the Cauchy-Riemann equations (\ref{creqex}) at $(0,0)$.
\begin{align*}
u_x(0,0) &= \lim_{s \to 0}\frac{u(s,0) - u(0,0)}{s} = \lim_{s \to 0}\frac{\dfrac{s^3}{s^2} - 0}{s} = 1\\[0.5em]
u_y(0,0) &= \lim_{t \to 0}\frac{u(0,t) - u(0,0)}{t} = \lim_{t \to 0}\frac{0 - 0}{t} = 0  \\[1em]
v_x(0,0) &= \lim_{s \to 0}\frac{v(s,0) - v(0,0)}{s} = \lim_{s \to 0}\frac{0 - 0}{s} = 0\\[0.5em]
v_y(0,0) &= \lim_{t \to 0}\frac{v(0,t) - v(0,0)}{t} = \lim_{t \to 0}\frac{\dfrac{t^3}{t^2} - 0}{t} = 1
\end{align*}
Therefore $u_x(0,0) = 1 = v_y(0,0)$ and $u_y(0,0) = 0 = -v_x(0,0)$, and hence the Cauchy-Riemann equations (\ref{creqex}) are satisfied. But $f'(0)$ does not exist, as seen in Problem \ref{prob 6.6}.
\end{example}

\vspace*{1em}

Imposing certain existence and continuity conditions on the first order partial derivatives of $u$ and $v$, the Cauchy-Riemann equations (\ref{creqex}) can be upgraded to a sufficient condition for differentiability.
\begin{theorem}[Sufficient Conditions for Differentiability]\label{crsuff}
Consider a function \[f(z) = f(x + iy) = u(x,y) + i\,v(x,y)\]
and a $z_0$ in the domain of $f$, such that
\begin{itemize}
\item[(a)] the first order partial derivatives of $u$ and $v$ exist and are continuous in an open disk centered at $z_0$; and
\item[(a)] the Cauchy-Riemann equations (\ref{creqex}) are satisfied at $(x_0,y_0)$.
\end{itemize}
Then $f'(z_0)$ exists and is given by $u_x(x_0,y_0) + i\,v_x(x_0,y_0) = v_y(x_0,y_0)-i\,u_y(x_0,y_0)$.
\end{theorem}
\begin{proof}
We skip the proof. You can find a proof in [1, Section 22, Page 66].
\end{proof}

\vspace*{1.5em}

\begin{example}
Let's revisit examples from Example \ref{necessarycr} and \ref{onlynecescr}.
\begin{itemize}
\item[(1)] Consider $f(z) = \abs{z}^2 = x^2 + y^2$, we noted that $u(x,y) = x^2 + y^2$ and $v(x,y) = 0$. We have seen that the only point where $f(z)$ can be differentiable is $z = 0$. The partial derivatives in a neighbourhood of $(0,0)$ are
\begin{align*}
u_x &= 2x & v_x &= 0\\[0.5em]
u_y &= 2y & v_y &= 0
\end{align*}
which clearly exist and are continuous. We have also seen that the Cauchy-Riemann equations (\ref{creqex}) are satisfied at $(0,0)$, trivially. Therefore $f'(0)$ exists and \[f'(0) = u_x(0,0) + i\,v_x(0,0) = 0.\]

\item[(2)] Consider $f(z) = (z + i\overline{z})^2$, we noted that $u(x,y) = 0$ and $v(x,y) = 2(x + y)^2$. We have seen that the only point where $f(z)$ can be differentiable are $z = x + iy \in \cc$ such that $y = \Im z = -\Re z = -x$. That is, at points of the form $(x,-x)$. The partial derivatives in a neighbourhood of $(x,-x)$ are
\begin{align*}
u_x &= 0 & v_x &= 4(x + y)\\[0.5em]
u_y &= 0 & v_y &= 4(x + y)
\end{align*}
which clearly exist and are continuous. Note the Cauchy-Riemann equations (\ref{creqex}) are satisfied at $(x,-x)$ trivially, since \[u_x(x,-x) = u_y(x,-x) = v_x(x,-x) = v_y(x,-x) = 0.\]
Therefore $f'(z)$ exists, for $z = x - ix$, and \[f'(z) = u_x(x,-x) + i\,v_x(x,-x) = 0.\]

\item[(3)] The reason Example \ref{onlynecescr} doesn't contradict Theorem \ref{crsuff} is because, $u_x$, in particular, is not continuous at $(0,0)$. Note that we have
\[u(x,y) = \begin{cases}\dfrac{x^3 - 3xy^2}{x^2 + y^2} & (x,y) \neq (0,0)\\[1em] 0 & (x,y) = (0,0) \end{cases}\]
For $(x,y) \neq (0,0)$, we compute $u_x(x,y)$ using the quotient rule, while we have already computed $u_x(0,0) = 1$ in Example \ref{onlynecescr}, giving us
\[u_x(x,y) = \begin{cases}\dfrac{x^4 + 6x^2y^2 - 3y^4}{(x^2 + y^2)^2} & (x,y) \neq (0,0)\\[1em] 1 & (x,y) = (0,0) \end{cases}\]
Suppose $u_x(x,y)$ is continuous at $(0,0)$, then we have
\[\lim_{(x,y) \to (0,0)}u_x(x,y) = \lim_{(x,y) \to (0,0)}\dfrac{x^4 + 6x^2y^2 - 3y^4}{(x^2 + y^2)^2} = u_x(0,0) = 1\]
Restricting the limit along the $y$-axis, where $x = 0$, we get
\[1 = \lim_{(0,y) \to (0,0)}\dfrac{-3y^4}{(y^2)^2} = \lim_{y \to 0}\dfrac{-3y^4}{y^4} = -3,\]
a contradiction. Hence, $u_x(x,y)$ is not continuous at $(0,0)$.
\end{itemize}
\end{example}

\vspace*{1em}

\begin{example}[Complex Exponential]\label{expcmplxeg}
Define, for any $z = x + iy \in \cc$
\[\exp(z) = e^z \coloneqq e^xe^{iy} = e^x(\cos y + i\,\sin y)\]
the \emph{complex exponential function}. Note that $e^x$ is the usual real exponential and $e^{iy}$ is given by Euler's formula (Definition \ref{eulerform}). Here, 
\[u(x,y) = e^x\cos y \quad \text{and} \quad v(x,y) = e^x\sin y\]
We then see that
\begin{align*}
u_x &= e^x\cos y = v_y,\\[0.5em] v_x &= -e^x\sin y = -u_y;
\end{align*}
so $\exp$ satisfies the Cauchy-Riemann equations (\ref{creqex}) everywhere. Furthermore, $u_x,\,u_y,\,v_x$ and $v_y$ are everywhere defined and continuous. Hence $\exp$ is everywhere complex-differentiable, an \emph{entire} function. Furthermore $\exp(z)' = u_x + iv_x = e^x\cos y + ie^x\sin y = \exp(z)$.
\end{example}

\vspace*{1em}

\begin{discussion}[Polar Cauchy-Riemann Equations]
Recall that if the domain of a function $f$ is contained in $\cc^*$ or restricted to within $\cc^*$, one can express in polar coordinates at $z = re^{i\theta}$ as
\[f(z) = f(re^{i\theta}) = u(r,\theta) + i\,v(r,\theta)\]
Then, the Cauchy-Riemann equations (\ref{creqex}) at a point $(r_0,\theta_0)$ can be expressed in polar coordinates, \cdef{Polar\ Cauchy\text{-}Riemann\ Equations} (see Problem \ref{prob 7.3})
\begin{align*}\label{pcreqex}
ru_r &= v_\theta\\[-0.5em]
\tag{Polar CR}\\[-0.5em]
u_\theta &= -rv_r
\end{align*}
and a differentiable function at $z_0 = r_0e^{i\theta_0}$ is then expressed as \[f'(z_0) = f'(r_0e^{i\theta_0}) = e^{-i\theta_0}(u_r(r_0,\theta_0) + i\,v_r(r_0,\theta_0)).\]
\end{discussion}

\vspace*{1em}

\begin{example}
Consider the function
\[f(z) = f(re^{i\theta}) = \sqrt{r}\,e^{i\frac{\theta}{2}} ,\]
where $r > 0$ and $-\pi < \theta < \pi$. This is the function that outputs the principal square root of $z$. We compute $f'(z)$ at $z = re^{i\theta}$ using the polar form of Theorem \ref{crsuff}. We first note that
\[f(z) = \underbrace{\sqrt{r}\cos\left(\frac{\theta}{2}\right)}_{u(r,\theta)} + i\underbrace{\sqrt{r}\sin\left(\frac{\theta}{2}\right)}_{v(r,\theta)}\]
Now, we compute
\begin{align*}
ru_r &= r\frac{1}{2\sqrt{r}}\cos\left(\frac{\theta}{2}\right) = \frac{\sqrt{r}}{2}\cos\left(\frac{\theta}{2}\right) = v_\theta\\[0.5em]
u_\theta &= -\frac{\sqrt{r}}{2}\sin\left(\frac{\theta}{2}\right) = -r\frac{1}{2\sqrt{r}}\sin\left(\frac{\theta}{2}\right) = -rv_r
\end{align*}
Clearly the first order partial derivatives exist everywhere and the Polar Cauchy-Riemann equations (\ref{pcreqex}) are also satisfied everywhere. Hence $f'(z)$ exists and
\begin{align*}
f'(z) &= e^{-i\theta}(u_r(r,\theta) + i\,v_r(r,\theta))\\[0.5em]
&= e^{-i\theta}\left(\frac{1}{2\sqrt{r}}\cos\left(\frac{\theta}{2}\right) + i\,\frac{1}{2\sqrt{r}}\sin\left(\frac{\theta}{2}\right)\right)\\[0.5em]
&= \frac{e^{-i\theta}}{2\sqrt{r}}\left(\cos\left(\frac{\theta}{2}\right) + i\,\sin\left(\frac{\theta}{2}\right)\right)\\[0.5em]
&= \frac{1}{2\sqrt{r}}\cdot e^{-i\theta}\cdot e^{i\frac{\theta}{2}}\\[0.5em]
&= \frac{1}{2\sqrt{r}e^{i\frac{\theta}{2}}}\\[0.5em]
&= \frac{1}{2f(z)}
\end{align*}
\end{example}

\vspace*{2em}

\subsection{Problems}
\vspace{0.1in}

\begin{problem}\label{prob 7.1}
Define 
\[f(z) = \begin{cases} 0 & \text{if }\Re(z)\cdot \Im(z) = 0,\\[0.5em] 1 & \text{if } \Re(z)\cdot \Im(z) \neq 0\end{cases}.\]
Show that $f$ satisfies the Cauchy–Riemann equation at $z = 0$, yet $f$ is not diferentiable at $z = 0$.
\end{problem}

\vspace{0.1in}

\begin{problem}\label{prob 7.2}
Show that when $f(z) = x^3 + i(1 - y)^3$, it makes sense to write
\[f'(z) = u_x + iv_x = 3x^2\]
only when $z = i$.
\end{problem}

\vspace{0.1in}

\begin{problem}\label{prob 7.3}
Show that $f'(z)$ does not exist at any point if
\begin{itemize}
\item[(a)] $f(z) = z - \overline{z}$
\item[(b)] $f(z) = 2x + ixy^2$
\end{itemize}
\end{problem}

\vspace{0.1in}

\begin{problem}\label{prob 7.4}
Show that $f'(z)$ and its derivative $f''(z)$ exist everywhere, and find $f''(z)$ when
\begin{itemize}
\item[(a)] $f(z) = iz + 2$
\item[(b)] $f (z) = e^{-x}e^{-iy}$
\end{itemize}
\end{problem}

\vspace{0.1in}

\begin{problem}\label{prob 7.5}
Let $f: G \to \cc$ be a function, such that $G \subseteq \cc^*$, then we can write
\[f(z) = f(x + iy) = u(x,y) + i\,v(x,y)\quad \text{or}\quad f(z) = f(re^{i\theta}) = u(r,\theta) + i\,v(r,\theta)\]
Using the fact that $x = r\cos\theta$ and $y = r\sin\theta$ and the chain rule from calculus, write $u_r$ and $u_\theta$ in terms of $u_x$ and $u_y$. Assuming $f$ is differentiable, rewrite the \ref{creqex}-equations and $f'(z)$ in terms of $u_r$ and $u_\theta$.
\end{problem}

\vspace{0.1in}

\begin{problem}\label{prob 7.6}
Prove that the function
\[f(z) = e^{-\theta}\cos(\ln r) + ie^{-\theta}\sin(\ln r)\]
is differentiable when $r > 0$ and $0 < \theta < 2\pi$, and find $f'(z)$ in terms of $f(z)$.
\end{problem}