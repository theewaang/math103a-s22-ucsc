\vspace*{1em}

\begin{mdframed}
\begin{center}
{\Large Continuous Functions}
\end{center}
\end{mdframed}

\begin{definition}[Continuous Functions]
A function $f: G \to \cc$ is \emph{continuous at $z_0 \in G$} if either $z_0$ is an isolated point or 
\[\lim_{z \to z_0}\,f(z) = f(z_0) = f\left(\lim_{z\to z_0}\,z\right)\]
That is, for all $\epsilon > 0$, there exists a $\delta > 0$ such that
\[\text{if }\ 0 < \abs{z - z_0} < \delta,\quad \text{then }\ \abs{f(z) - f(z_0)}<\epsilon.\]
A function is \cdef{continuous} if it is continuous at every point in its domain.\\
\\
By the limit laws (Theorem \ref{limlaw}), sum, product and quotient of continuous functions are continuous (whenever and wherever defined).
\end{definition}

\vspace*{1em}

\begin{theorem}[Composition of Continuous Functions]\label{composcont}
Suppose we have two functions $f:G_1 \to \cc$ and $g: G_2 \to \cc$ such that $f(G_1) \subseteq G_2$. If $f$ is continuous at $z_0$ and $g$ is continuous at $f(z_0)$, then $g\circ f$ is continuous at $z_0$. That is,
\[\lim_{z \to z_0}g(f(z)) = g(f(z_0)) = g\left(\lim_{z\to z_0}f(z)\right) = g\left(f\left(\lim_{z\to z_0}\,z\right)\right)\]
Therefore, if $f$ and $g$ are continuous, so is $g\circ f$.
\end{theorem}
\begin{proof}
By continuity of $g$ at $f(z_0)$, for an arbitrary $\epsilon > 0$, there exists a $\delta_1 > 0$ such that
\[\text{if }\ 0 < \abs{w - f(z_0)} < \delta_1,\quad \text{then }\ \abs{g(w) - g(f(z_0))}<\epsilon.\]
Now, by continuity of $f$ at $z_0$, for $\delta_1 > 0$, there exists a $\delta > 0$ such that
\[\text{if }\ 0 < \abs{z - z_0} < \delta,\quad \text{then }\ \abs{f(z) - f(z_0)}<\delta_1.\]
With these two statements, we have that 
\[\text{if }\ 0 < \abs{z - z_0} < \delta,\quad \text{then }\ \abs{g(f(z)) - g(f(z_0))}<\epsilon.\]
Therefore $g\circ f$ is continuous at $z_0$.
\end{proof}

\vspace*{1em}

\begin{theorem}
Suppose $f: G \to \cc$ is continuous at $z_0$ and $f(z_0) \neq 0$, then there exists a $\delta > 0$ such that $f(z) \neq 0$ for all $z \in D_\delta(z_0)$. That is, $\abs{f(z)} > 0$ for all $z \in D_\delta(z_0)$.
\end{theorem}
\begin{proof}
Since $f$ is continuous and non-zero at $z_0$, for $\epsilon = \dfrac{\abs{f(z_0)}}{2} > 0$ there exists a $\delta > 0$ such that
\[\text{if }\ z \in D_\delta(z_0),\quad \text{then }\ \abs{f(z) - f(z_0)}< \frac{\abs{f(z_0)}}{2}.\]
For such a $z$, the reverse triangle inequality gives us
\[\abs{\abs{f(z)} - \abs{f(z_0)}} \leq \abs{f(z) - f(z_0)}< \frac{\abs{f(z_0)}}{2};\quad \text{so, }\ -\frac{\abs{f(z_0)}}{2} < \abs{f(z)} - \abs{f(z_0)} < \frac{\abs{f(z_0)}}{2}\]
since the former is the absolute value of real numbers. Therefore, adding $\abs{f(z_0)}$ to this inequality gives us
\[\abs{f(z)} > \frac{\abs{f(z_0)}}{2} > 0\]
as needed.
\end{proof}

\vspace*{1em}

\begin{theorem}[Continuity in terms of Real and Imaginary parts of a Function]\label{contpart}
Suppose that 
\[f(z) = f(x + iy) = u(x,y) + i\,v(x,y).\]
Then $f$ is continuous at $z_0 = x_0 + iy_0$ if and only if $u$ and $v$ are continuous at $(x_0,y_0)$.
\end{theorem}
\begin{proof}
This is directly follows from Theorem \ref{cmplxlimripart}.
\end{proof}

\vspace*{1em}

\begin{definition}[Compact Sets]
A subset of $\cc$ is said to be \cdef{compact} if it is closed and bounded.
\end{definition}

\vspace*{1em}

\begin{definition}[Bounded Functions]
A function $f: G \to \cc$ is said to be a \cdef{bounded\ function} if the image $f(G)$ is bounded. Equivalently, if there exists $M > 0$ such that $\abs{f(z)} \leq M$ for every $z \in G$.
\end{definition}

\vspace*{1em}

\begin{theorem}[Extreme Value Theorem]\label{evt}
Suppose $K \subseteq \cc$ is compact, and $f: K \to \cc$ is continuous. Then $f$ is bounded, that is there exists an $M > 0$ such that $\abs{f(z)} \leq M$ for all $z \in K$, and there exists a $z_0 \in K$ such that $\abs{f(z_0)} = M$.
\end{theorem}
\begin{proof}
Since $f = u + iv$ is continuous, so are $u,v:\rr^2 \to \rr$ by Theorem \ref{contpart}.  Hence, so is
\[\abs{f(z)} = \abs{f(x + iy)} = \sqrt{u(x,y)^2 + v(x,y)^2}\]
as it's obtained as a sum, product and composition of continuous functions. This result then follows from standard Calculus, since $\abs{f}$ is a real-valued function.
\end{proof}

\vspace*{2em}

\begin{mdframed}
\begin{center}
{\Large Complex-Differentiable Functions}
\end{center}
\end{mdframed}

\begin{definition}[Derivative]\label{cmplxder}
Consider a function $f:G \to \cc$, the \cdef{derivative} \emph{of $f$ at $z_0 \in G$} is the limit
\[\frac{d}{dz}(f(z_0)) = f'(z_0) \coloneqq \lim_{z \to z_0}\frac{f(z) - f(z_0)}{z - z_0}\]
If the limit exists, we say $f$ is \emph{differentiable at $z_0$}.\\[0.5em]
A function is \cdef{differentiable} if it is differentiable at every point in its domain.
\end{definition}
Letting $h = \Delta_{z_0}z = z - z_0$, the limit can also be written as
\[f'(z_0) = \lim_{h \to 0}\frac{f(z_0 + h) - f(z_0)}{h}\]

\vspace*{1em}

\begin{example}
Consider $f(z) = z^2$, then
\begin{align*}
f'(z) = \lim_{h \to 0}\frac{f(z + h) - f(z)}{h} &= \lim_{h \to 0}\frac{(z + h)^2 - z^2}{h}\\[0.5em]
&= \lim_{h \to 0}\frac{2zh + h^2}{h}\\[0.5em]
&= \lim_{h \to 0}\,2z + h\\[0.5em]
&= 2z
\end{align*}
\end{example}

\vspace*{1em}

\begin{example}\label{normdiffexistence}
Where is $f(z) = \abs{z}^2$ differentiable?\\[1em]
Consider $z \in \cc$ and an arbitrary $h \in \cc$, then we compute
\begin{align*}
f(z + h) - f(z) &= \abs{z + h}^2 - \abs{z}^2\\[0.5em]
&= (z + h)\overline{(z + h)} - z\overline{z}\\[0.5em]
&= z\overline{z} + z\overline{h} + \overline{z}h + h\overline{h} - z\overline{z}\\[0.5em]
&= z\overline{h} + \overline{z}h + h\overline{h}
\end{align*}
Then 
\[\frac{f(z + h) - f(z)}{h} = \frac{z\overline{h} + \overline{z}h + h\overline{h}}{h} = z\,\frac{\overline{h}}{h} + \overline{z} + \overline{h}\]
Along the real axis, $h = \overline{h}$, we have
\[\frac{f(z + h) - f(z)}{h} = z + \overline{z} + h;\]
therefore, as $h \to 0$, the limit is $z + \overline{z}$. Along the imaginary axis, $h = -\overline{h}$, we have
\[\frac{f(z + h) - f(z)}{h} = -z + \overline{z} - h;\]
therefore, as $h \to 0$, the limit is $-z + \overline{z}$.\\
\\
Since limits are unique, if $f'(z)$ exists, then $z + \overline{z} = -z + \overline{z}$, which gives us $z = 0$. That is, if $f'(z)$ exists, it only exists for $z = 0$. So, does $f'(0)$ exist?
\[f'(0) = \lim_{h \to 0}\frac{f(h) - f(0)}{h} = \lim_{h \to 0}\frac{h\overline{h}}{h} = \lim_{h \to 0}\overline{h} = 0\]
\end{example}

\vspace*{1em}

\begin{proposition}[Differentiable Functions are Continuous]
If $f$ is differentiable at $z_0$, then $f$ is continuous at $z_0$.
\end{proposition}
\begin{proof}
Suppose $f$ is differentiable at $z_0$, then
\[\lim_{z \to z_0}f(z) - f(z_0) = \left(\lim_{z \to z_0}\frac{f(z) - f(z_0)}{z - z_0}\right)\left(\lim_{z \to z_0}\,z - z_0\right) = f'(z_0)\cdot 0 = 0\]
Therefore $\lim_{z \to z_0}f(z) = f(z_0)$, and hence $f$ is continuous at $z_0$.
\end{proof}

\vspace*{1em}

\begin{theorem}[Differentiation Laws]
Suppose $f$ and $g$ are differentiable at $z$. Then,
\begin{itemize}[itemsep=1em]
\item[(1)] $(c)' = 0$, for every $c \in \cc$.
\item[(2)] $(c\cdot f)'(z) = c\cdot f'(z)$, for every $c \in \cc$.\hfill \emph{(Constant Rule)}
\item[(3)] $(z^n)' = nz^{n-1}$, for every $n \in \zz$ (assume $z \neq 0$ for $n<0$).\hfill \emph{(Power Rule)}
\item[(4)] $(f + g)'(z) = f'(z) + g'(z)$.\hfill \emph{(Sum Rule)}
\item[(5)] $(fg)'(z) = f'(z)g(z) + f(z)g'(z)$.\hfill \emph{(Product Rule)}
\item[(6)] $\left(\dfrac{f}{g}\right)'(z) = \dfrac{f'(z)g(z) - f(z)g'(z)}{g(z)^2}$, provided $g(z) \neq 0$ \hfill \emph{(Quotient Rule)}
\end{itemize}
\end{theorem}
\begin{proof}
(1) and (4) are proved directly using the limit definition, (2) can be proved directly or using (1) and (5), while (3) can be proven inductively using (5) for positive $n$ and (6) for negative $n$. 
\begin{itemize}
\item[(5)] We first compute
\begin{align*}
f(z + h)g(z + h) - f(z)g(z) &= f(z + h)g(z + h) - f(z)g(z) + f(z + h)g(z) - f(z + h)g(z)\\[0.5em]
&= f(z + h)(g(z + h) - g(z)) + g(z)(f(z + h) - f(z))
\end{align*}
So,
\begin{align*}
(fg)'(z) &= \lim_{h \to 0}\frac{f(z + h)g(z + h) - f(z)g(z)}{h}\\[0.5em]
&= \lim_{h \to 0}\frac{f(z + h)(g(z + h) - g(z))}{h} + \lim_{h \to 0}\frac{g(z)(f(z + h) - f(z))}{h}\\[0.5em]
&= \lim_{h \to 0}f(z + h)\cdot\lim_{h \to 0}\frac{g(z + h) - g(z)}{h} + g(z)\lim_{h \to 0}\frac{f(z + h) - f(z)}{h}\\[0.5em]
&= f(z)g'(z) + g(z)f'(z)
\end{align*}
\item[(6)] We first compute
\begin{align*}
\frac{1}{g(z + h)} - \frac{1}{g(z)} &= \frac{g(z) - g(z + h)}{g(z)g(z + h)}\\[0.5em]
&= -\frac{g(z+h) - g(z)}{g(z)g(z + h)}
\end{align*}
So,
\begin{align*}
\left(\dfrac{1}{g}\right)'(z) &= \lim_{h \to 0}\frac{\dfrac{1}{g(z + h)} - \dfrac{1}{g(z)}}{h}\\[0.5em]
&= \lim_{h \to 0}-\frac{g(z+h) - g(z)}{g(z)g(z + h)}\cdot\frac{1}{h}\\[0.5em]
&= -\lim_{h \to 0}\frac{g(z+h) - g(z)}{h}\cdot\lim_{h \to 0}\frac{1}{g(z)g(z + h)}\\[0.5em]
&= -\frac{g'(z)}{g(z)^2}
\end{align*}
(6) then follows from the computation above and using (5) on $\dfrac{f(z)}{g(z)} = f(z)\cdot\dfrac{1}{g(z)}$.
\end{itemize}
\vspace*{-\baselineskip}
\end{proof}

\vspace*{1em}

\begin{proposition}[Chain Rule]
Suppose we have two functions $f:G_1 \to \cc$ and $g: G_2 \to \cc$ such that $f(G_1) \subseteq G_2$. If $f$ is differentiable at $z_0$ and $g$ is differentiable at $f(z_0)$, then $g\circ f$ is differentiable at $z_0$ and
\[(g\circ f)'(z_0) = g'(f(z_0))\cdot f'(z_0)\]
\end{proposition}
\begin{proof}
Let's start by defining an auxiliary function on $G_2$
\[\phi(w) = \begin{cases} \dfrac{g(w) - g(f(z_0))}{w - f(z_0)} - g'(f(z_0)) & w \neq f(z_0)\\[0.5em] 0 & w = f(z_0) \end{cases}\]
Since $g$ is differentiable at $f(z_0)$, then $\lim_{w \to f(z_0)}\phi(w) = 0 = \phi(f(z_0))$ and therefore $\phi$ is continuous at $f(z_0)$. Furthermore, since $f$ is differentiable at $z_0$, it is continuous at $z_0$. So $\lim_{z \to z_0}\phi(f(z)) = \phi(f(z_0)) = 0$ by Theorem \ref{composcont}.\\[1em]
Rewriting the above expression, we get the following expression which is valid on all of $G_2$.
\[g(w) - g(f(z_0)) = (w - f(z_0))(\phi(w) + g'(f(z_0)))\]
Now, for $w = f(z) \in f(G_1)$, we have
\begin{align*}
\frac{g(f(z)) - g(f(z_0))}{z - z_0} &= \frac{(f(z) - f(z_0))(\phi(f(z)) + g'(f(z_0)))}{z - z_0}\\[0.5em]
&= (\phi(f(z)) + g'(f(z_0)))\cdot\frac{f(z) - f(z_0)}{z - z_0}
\end{align*}
Therefore, 
\begin{align*}
(g\circ f)'(z_0) &= \lim_{z \to z_0}\frac{g(f(z)) - g(f(z_0))}{z - z_0}\\[0.5em]
&= \lim_{z \to z_0}(\phi(f(z)) + g'(f(z_0)))\cdot\lim_{z \to z_0}\frac{f(z) - f(z_0)}{z - z_0}\\[1em]
&= g'(f(z_0))\cdot f'(z_0),\ \text{since $\lim_{z \to z_0}\phi(f(z)) = 0$}
\end{align*}
\end{proof}

\vspace*{0.2in}

\subsection{Problems}
\vspace{0.1in}

\begin{problem}\label{prob 6.1}
Example \ref{polycts} tells us that polynomials are continuous. 
\begin{itemize}
\item[(a)] Prove that the complex conjugation function $\sigma(z) \coloneqq \overline{z}$ is continuous.
\item[(b)] Prove that a polynomial in $\overline{z}$ is continuous. That is, prove that a polynomial given as
\[p(\overline{z}) = a_n\overline{z}^n + \cdots + a_1\overline{z} + a_0,\quad a_i \in \cc,\ a_n \neq 0\]
is continuous.
\item[(c)] Prove that the following functions are continuous by writing them as a sum or product of polynomials $p(z)$ and $q(\overline{z})$
\begin{itemize}
\item[(i)] $R(z) \coloneqq \Re z$
\item[(ii)] $I(z) \coloneqq \Im z$
\item[(iii)] $N(z) \coloneqq \abs{z}^2$
\end{itemize}
\end{itemize}
\end{problem}

\vspace{0.1in}

\begin{problem}\label{prob 6.2}
Show that the function $f : \cc \to \cc$ given by
\[f(z) = \begin{cases} \dfrac{\overline{z}}{z} & \text{if }z \neq 0\\[1em] 1 & \text{if } z = 0 \end{cases}\]
is continuous on $\cc^*$.
\end{problem}

\vspace{0.1in}

\begin{problem}\label{prob 6.3}
Consider the function \[f : \cc^*\to \cc,\ z \mapsto \frac{1}{z}.\]
Apply the definition of the derivative to give a direct proof that $f'(z) = -\dfrac{1}{z^2}$.
\end{problem}

\vspace{0.1in}

\begin{problem}\label{prob 6.4}
Find the derivative of the function 
\[M(z) \coloneqq \frac{az + b}{cz + d},\quad ad-bc \neq 0.\]
When is $M'(z) = 0$?
\end{problem}

\vspace{0.1in}

\begin{problem}\label{prob 6.5}
Using Example \ref{limnotex} as an inspiration, show that $f'(z)$ does not exist for any $z$ for the functions
\begin{itemize}
\item[(a)] $f(z) = \Re z$
\item[(b)] $f(z) = \Im z$
\end{itemize}
\end{problem}

\vspace{0.1in}

\begin{problem}\label{prob 6.6}
Show that the function $f : \cc \to \cc$ given by
\[f(z) = \begin{cases} \dfrac{\overline{z}^2}{z} & \text{if }z \neq 0\\[1em] 0 & \text{if } z = 0 \end{cases}\]
is not differentiable at $0$.
\end{problem}

\vspace{0.1in}

\begin{problem}\label{prob 6.7}\hfill
\begin{itemize}
\item[(a)] Show that a polynomial of degree $n$, $p(z) = a_0 + a_1z + a_2z^2 + \cdots + a_nz^n$, where $a_n \neq 0$, is differentiable everywhere, with
\[p'(z) = a_1 + 2a_2z + \cdots + na_nz^{n-1}\]
\item[(b)] Furthermore, show that for $p(z)$, as given in (a), we have
\[a_i = \frac{p^{(i)}(0)}{i!}\]
for $i = 0,\ldots,n$. Where $p^{(0)}(z) = p(z)$ and $p^{(i)}(z)$, for $i>0$, is the $i^{\text{th}}$ derivative of $p(z)$.
\end{itemize}
\end{problem}

\vspace{0.1in}

\begin{problem}\label{prob 6.8}
Let $G$ be a domain and $f: G \to \cc$ a function that is differentiable at every point in $G$. Consider the domain
\[G^* = \setp{z \in \cc}{\overline{z} \in G}\]
and the function 
\[f^*:G^* \to \cc,\ z \mapsto \overline{f(\overline{z})}\]
Show that $f^*$ is differentiable at every point in $G^*$.
\end{problem}

\vspace{0.1in}

\begin{problem}\label{prob 6.9}
For each function, determine all points at which the derivative exists. When the derivative exists, find its value. Use Example \ref{normdiffexistence} as an inspiration.
\begin{itemize}
\item[(a)] $f(z) = z + i\overline{z}$
\item[(b)] $g(z) = (z + i\overline{z})^2$
\item[(b)] $h(z) = z\Im z$
\end{itemize}
\end{problem}

\vspace{0.1in}

\begin{problem}\label{prob 6.10}
By definition, a function $f : G \to \cc$ is differentiable at $z_0 \in G$ if the limit
\[f'(z_0) = \lim_{z\to z_0}\frac{f(z) - f(z_0)}{z - z_0}\]
exists. Unpacking the limit definition, we see that $f$ is differentiable at $z_0$ if and only if for every $\epsilon > 0$, there exists a $\delta > 0$ such that
\[\text{if }\ 0 < \abs{z - z_0} < \delta,\quad \text{then }\ \abs{\frac{f(z) - f(z_0)}{z - z_0} - f'(z_0)} < \epsilon.\]
\newpage
By appealing only to the definition, we show that $\sigma : \cc \to \cc$ defined by $\sigma(z) = \overline{z}$ is not differentiable anywhere by completing the following steps.
\begin{itemize}
\item[(i)] Let $z_0 \in \cc$ and assume that $f'(z_0)$ exists. Choose $\delta > 0$ according to the definition using $\epsilon = 1/2$ and write down the resulting statement.
\item[(ii)] Consider $z = z_0 + \delta/2$ and conclude from (a) that $\abs{1 - f'(z_0)} < \epsilon$.
\item[(iii)] Consider $z = z_0 + i\delta/2$ and conclude from (a) that $\abs{1 + f'(z_0)} < \epsilon$.
\item[(iv)] Using the triangle inequality together with (ii) and (iii), obtain a contradiction.
\end{itemize}
\end{problem}