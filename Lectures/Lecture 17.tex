\vspace*{1em}

\begin{theorem}[Morera's Theorem]\label{morera}
Suppose $f$ is continuous on a domain $G$. If
\[\int_C\,f(z)\,dz = 0\]
for every closed contour $G$, then $f$ is holomorphic on $G$. 
\end{theorem}
\begin{proof}
By Theorem \ref{FTCoCI}, there exists a holomorphic function $F:G \to \cc$ such that $F'(z) = f(z)$ for all $z \in G$. But by Theorem \ref{holsmooth}, $F'$ is holomorphic on $G$, and therefore so is $f$. 
\end{proof}

\vspace*{1em}

\begin{remark}
When $G$ is simply connected, Morera's theorem (Theorem \ref{morera}) is just the converse of Cauchy-Goursat Theorem for simply connected domains (Theorem \ref{cgthmsc}).
\end{remark}

\vspace*{1em}

\begin{theorem}[Cauchy's Inequalities]\label{cauchyineq}
Suppose that $f$ is holomorphic on all points on and interior to $C_R = C_R(z_0)$, a positively oriented circle of radius $R$ centered at some $z_0 \in \cc$. Then,
\[|f^{(n)}(z_0)| \leq \frac{n!}{R^n}\,\max_{z \in C_R(z_0)}\abs{f(z)}\]
\end{theorem}
\begin{proof}
By Theorem \ref{gencintform}, 
\[f^{(n)}(z_0) = \frac{n!}{2\pi i}\int_{C_R}\,\frac{f(z)}{(z - z_0)^{n+1}}\,dz\]
Hence, 
\begin{align*}
|f^{(n)}(z_0)|= \abs{\frac{n!}{2\pi i}\int_{C_R}\,\frac{f(z)}{(z - z_0)^{n+1}}\,dz} &= \frac{n!}{2\pi i}\abs{\int_{C_R}\,\frac{f(z)}{(z - z_0)^{n+1}}\,dz}\\[1em]
&\leq \frac{n!}{2\pi i}\max_{z \in C_R}\,\abs{\frac{f(z)}{(z - z_0)^{n+1}}}\cdot L(C_R)\\[1em]
&= \frac{n!}{2\pi i}\max_{z \in C_R}\,\frac{\abs{f(z)}}{R^{n+1}}\cdot 2\pi R\\[1em]
&= \frac{n!}{R^n}\,\max_{z \in C_R}\abs{f(z)}\\[-2.5em]
\end{align*}
\end{proof}

\vspace*{2em}

\begin{mdframed}
\begin{center}
{\Large Liouville's Theorem and the Fundamental Theorem of Algebra}
\end{center}
\end{mdframed}
As an application, we will prove that every non-constant polynomial with complex coefficients has a root in $\cc$. In the language of algebra, we will provide a proof for the fact that $\cc$ is \emph{algebraically closed}. Thus, the the statement is ``purely algebraic" while no ``purely algebraic" proof exists. The proof relies on the following wonderful theorem. 

\vspace*{1em}

\begin{theorem}[Liouville's Theorem]\label{liouville}
Every bounded entire function is constant. 
\end{theorem}
\begin{proof}
We show that $f'(z) = 0$ for all $z \in \cc$, then it follows that $f$ is constant since $\cc$ is a domain by Theorem \ref{der0const}.\\
\\
Consider any $z_0 \in \cc$. Since $f$ is bounded, we can find a $M>0$ such that $\abs{f(z)} \leq M$ for all $z \in \cc$. Let $C_R(z_0)$ be a circle of radius $R$ centered at $z_0$, then $f$ is holomorphic at all points on and interior to $C_R(z_0)$. Hence, by Theorem \ref{cauchyineq},
\begin{align*}
\abs{f'(z_0)} &\leq \frac{1}{R}\max_{z \in C_R(z_0)}\abs{f(z)}\\[0.5em]
&\leq \frac{M}{R} \to 0,\ \text{ as } R \to \infty
\end{align*}
Thus $\abs{f'(z_0)} = 0$, giving us $f'(z_0) = 0$. Since $z_0$ was arbitrary, the result follows. 
\end{proof} 

\vspace*{1em}

\begin{theorem}[Fundamental Theorem of Algebra]
For any polynomial $p(z) = a_0 + a_1z + \cdots + a_nz^n$ where $a_n \neq 0,\ a_i \in \cc$ and $n \geq 1$, there exists an $\alpha \in \cc$ such that $p(\alpha) = 0$. That is, every non-constant polynomial $p(z)$ has at least one root in $\cc$. 
\end{theorem}
\begin{proof}
Suppose otherwise that $p(z)$ has no root in $\cc$, then $p(z) \neq 0$ for every $z \in \cc$. Hence $1/p(z)$ is an entire function. We show that $1/p(z)$ is bounded.\\
\\
For a non-zero $z \in \cc$, consider the complex number
\[w_z \coloneqq \frac{a_0}{z^n} + \frac{a_1}{z^{n-1}} + \cdots + \frac{a_{n-1}}{z}\]
Note that $p(z) = (w_z + a_n)\ z^n$, and by triangle inequality we have
\[\abs{w_z} \leq \frac{\abs{a_0}}{\abs{z}^n} + \frac{\abs{a_1}}{\abs{z}^{n-1}} + \cdots + \frac{\abs{a_{n-1}}}{\abs{z}} = \sum_{k=0}^{n-1}\frac{\abs{a_k}}{\abs{z}^{n-k}}\]
For each $0 \leq k \leq n-1$ we note that $\dfrac{\abs{a_k}}{\abs{z}^{n-k}} \to 0$ as $z \to \infty$.\\[0.5em]
Then, for $\epsilon = \dfrac{\abs{a_n}}{2n} > 0$, we can find an $R > 0$ such that whenever $\abs{z} > R$, we get
\[\frac{\abs{a_k}}{\abs{z}^{n-k}} = \abs{\frac{\abs{a_k}}{\abs{z}^{n-k}} - 0} < \epsilon = \frac{\abs{a_n}}{2n}\]
for any $k = 0,\ldots,n-1$. This then gives us
\[\abs{w_z} \leq \sum_{k=0}^{n-1}\frac{\abs{a_k}}{\abs{z}^{n-k}} < \sum_{k=0}^{n-1}\frac{\abs{a_n}}{2n} = n\cdot \frac{\abs{a_n}}{2n} = \frac{\abs{a_n}}{2}\]
Now, by the reverse triangle inequality we have
\[\abs{a_n + w_z} \geq \abs{\abs{a_n} - \abs{w_z}} > \abs{\abs{a_n} - \frac{\abs{a_n}}{2}} = \frac{\abs{a_n}}{2}\]
Thus, 
\begin{align*}
\abs{p(z)} &= \abs{(w_z + a_n)\ z^n}\\[0.5em]
&= \abs{w_z + a_n}\abs{z^n} > \frac{\abs{a_n}}{2}R^n
\end{align*}
Therefore, for any $z \in \cc$ such that $\abs{z} > R$, we have
\[\abs{\frac{1}{p(z)}} \leq \frac{2}{R^n\abs{a_n}}\]
So, $1/p(z)$ is bounded outside the closed disk $\overline{D}_R(0)$.\\[0.5em]
Now, the closed disk $\overline{D}_R(0)$ is compact (closed and bounded) and $1/p(z)$ is continuous on $\overline{D}_R(0)$. Hence $1/p(z)$ is bounded on $\overline{D}_R(0)$ by Theorem \ref{evt}.\\
\\
Thus, $1/p(z)$ is bounded on all of $\cc$. Hence, by Theorem \ref{liouville}, $1/p(z)$ is constant, and therefore so is $p(z)$. We have arrived a contradiction, since $p(z)$ was non-constant by assumption. 
\end{proof}

\vspace*{1em}

\begin{lemma}[Maximum Modulus Principle]
Suppose that $\abs{f(z)} \leq \abs{f(z_0)}$ at each point $z$ in a neighbourhood $D_\epsilon(z_0)$ where $f$ is holomorphic. Then $f(z) = f(z_0)$ on $D_\epsilon(z_0)$. That is, if a holomorphic function on an open disk achieves its maximum on it, then it is constant on the open disk.
\end{lemma}
\begin{proof}
Let $z_1 \in D_\epsilon(z_0)$ such that $z_1 \neq z_0$. Set $\rho \coloneqq \abs{z_1 - z_0} > 0$, and consider $C_\rho = C_\rho(z_0)$, the circle of radius $\rho > 0$ centered at $z_0$, which is interior to $D_\epsilon(z_0)$. We parametrise $C_\rho$ as $z(t) = z_0 + \rho e^{it}$ for $0 \leq t \leq 2\pi$.
\[
%\begin{tikzpicture}[scale=0.8]
%    \draw[<->,thick] (-1,0)--(5,0);
%	\draw[<->,thick] (0,-1)--(0,5);
%	\filldraw[teal,fill opacity=1/10,dashed](3,3) circle (2.5);
%    \draw[](3,3)--(5.5,3);
%    \fill (3,3) circle (2pt);
%    \node[] at (2.5,3) {$z_0$};
%    \node[] at (6,1.75) {$D_\epsilon(z_0)$};
%    \node[] at (4,3.4) {$\epsilon$};
%\end{tikzpicture}
\text{\color{red}add image here}
\]
By Theorem \ref{cintform},
\begin{align*}
\abs{f(z_0)} = \abs{\frac{1}{2\pi i}\int_{C_\rho}\,\frac{f(z)}{z - z_0}\, dz} &= \frac{1}{2\pi}\abs{\int_{C_\rho}\,\frac{f(z)}{z - z_0}\, dz}\\[1em]
 &= \frac{1}{2\pi}\abs{\int_0^{2\pi}\,\frac{f(z_0 + \rho e^{it})}{\rho e^{it}}\,i\rho e^{it}\ dt}\\[1em]
 &= \frac{1}{2\pi}\abs{\int_0^{2\pi}\,f(z_0 + \rho e^{it})\ dt}\\[1em]
 &\leq \frac{1}{2\pi}\int_0^{2\pi}\,|f(z_0 + \rho e^{it})|\ dt\\[1em]
 &\leq \frac{1}{2\pi}\int_0^{2\pi}\,\abs{f(z_0)}\ dt,\quad \text{by assumption}\\[1em]
 &\leq \abs{f(z_0)}
\end{align*} 
This tells us that
\[\abs{f(z_0)} = \frac{1}{2\pi}\int_0^{2\pi}\,|f(z_0 + \rho e^{it})|\ dt \tag{$\dagger$}\]
Since, $\displaystyle f(z_0) =  \frac{1}{2\pi}\int_0^{2\pi}\,\abs{f(z_0)}\ dt$. Rewriting ($\dagger$), we have
\[\frac{1}{2\pi}\int_0^{2\pi}\, \abs{f(z_0)} - |f(z_0 + \rho e^{it})|\ dt = 0\]
By assumption $\abs{f(z_0)} - |f(z_0 + \rho e^{it})| \geq 0$; suppose $\abs{f(z_0)} - |f(z_0 + \rho e^{it})| > 0$, then necessarily 
\[\frac{1}{2\pi}\int_0^{2\pi}\, \abs{f(z_0)} - |f(z_0 + \rho e^{it})|\ dt > 0\tag{$*$}\]
since the integrand in ($*$) is continuous in the variable $t$, giving us a contradiction. Thus, 
\[\abs{f(z_0)} - |f(z_0 + \rho e^{it})| = 0\]
Therefore $\abs{f(z)} = \abs{f(z_0)}$ for every $z \in C_\rho(z_0)$. Varying the radius $\rho > 0$, we may then obtain $\abs{f(z)} = \abs{f(z_0)}$ for every $z \in D_\epsilon(z_0)$.\\
\\
Thus, $\abs{f}$ is a holomorphic function on $D_\epsilon(z_0)$, and thus by Corollary \ref{absholconst}, we have $f$ is constant on $D_\epsilon(z_0)$ and $f(z) = f(z_0)$ for every $z \in D_\epsilon(z_0)$.
\end{proof}

\vspace*{2em}

\begin{mdframed}[backgroundcolor=paleyellow,linewidth=1pt]
\begin{center}
{\sc\Large Part IV. Series: Road to Residue Calculus}
\end{center}
\end{mdframed}

\begin{discussion}
We now begin discussing series of complex numbers; we assume results about real series from calculus. Among other things, this consideration leads to the following results.
\begin{itemize}[itemsep=1em]
\item[(1)] A function $f$ that is holomorphic on a disk $D_R(z_0)$ has a convergent power series expansion on that disk
\[f(z) = \sum_{k=0}^\infty\frac{f^{(k)}(z_0)}{k!}(z - z_0)^k\]
such functions are called \cdef{(complex)\ analytic}. Conversely, every power series (analytic function) $\sum_{k=0}^\infty a_k(z - z_0)^k$ is holomorphic. That is, a function is holomorphic if and only if it's analytic. 
\item[(2)] A function that's analytic on an annulus $R_1 < \abs{z - z_0} < R_2$ has a convergent series expansion on the annulus
\[f(z) = \sum_{k=0}^\infty a_k(z - z_0)^k + \sum_{k=1}^\infty\frac{b_k}{(z - z_0)^k}\]
with coefficients
\[a_k = \frac{1}{2\pi i}\int_C\frac{f(z)}{(z - z_0)^{k+1}}\,dz \quad \text{and} \quad b_k = \frac{1}{2\pi i}\int_C\frac{f(z)}{(z - z_0)^{-k+1}}\]
where $C$ is any positively oriented simple closed contour in the annulus and surrounding $z_0$. 
\end{itemize}
In fact, (2) provides a method for computing integrals over contours that surround a singular point! Suppose that $f$ has a singularity at $z_0$, but is holomorphic everywhere else on a deleted neighbourhood $D_R(z_0)\setminus\set{z_0}$. Then $f$ is holomorphic on the annulus $0 < \abs{z - z_0} < R$. If $C$ is any simple closed contour positive oriented around $z_0$ and lying inside the annulus, then according to (2), 
\[\int_C\, f(z)\,dz = 2\pi i\cdot b_1.\]
In other words, we can compute the contour integral of $f$ about a singularity just by computing the coefficient $b_1$ in the series expansion of $f$. This is the beginning of \emph{Calculus of Residues}.
\end{discussion}

\vspace*{2em}

\subsection{Problems}
\vspace{0.1in}
To be added
%\begin{problem}\label{prob 12.1}
%
%\end{problem}