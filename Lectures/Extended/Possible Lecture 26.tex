\vspace*{1em}

\begin{theorem}[Characterisation of Zeros]\label{zero-order}
Suppose that $f$ is holomorphic at $z_0 \in \cc$. Then the following are equivalent:
\begin{itemize}
\item[(1)] $z_0$ is a zero of $f$ of order $n$.
\item[(2)] $f(z) = (z - z_0)^ng(z)$ for some unique holomorphic function $g$ such that $g(z_0) \neq 0$.
\end{itemize}
\end{theorem}
\begin{proof}\hfill
\begin{itemize}[leftmargin=4.5em,itemsep=1.5em]
\item[(1) $\Rightarrow$ (2)] Let's assume $z_0$ is a zero of order $n \geq 1$, so $f^{(n)}(z_0) \neq 0$ and $f^{(k)}(z_0) = 0$ for $0 \leq k < n$. Since $f$ is holomorphic at $z_0$, it has a Taylor expansion on some disk $D_\epsilon(z_0)$
\begin{align*}
f(z) &= \sum_{k=0}^\infty \frac{f^{(k)}(z_0)}{k!}(z - z_0)^k\\[0.5em]
 &= \sum_{k=n}^\infty \frac{f^{(k)}(z_0)}{k!}(z - z_0)^k\\[0.5em]
 &= (z - z_0)^n\sum_{k=n}^\infty \frac{f^{(k)}(z_0)}{k!}(z - z_0)^{k - n}\\[0.5em]
 &= (z - z_0)^n\sum_{k=0}^\infty \frac{f^{(k + n)}(z_0)}{(k+n)!}(z - z_0)^k
\end{align*}
Define
\[g(z) = \sum_{k=0}^\infty \frac{f^{(k + n)}(z_0)}{(k+n)!}(z - z_0)^k\]
Clearly $g$ is holomorphic at $z_0$, since it converges on $D_\epsilon(z_0)$. Moreover,
\[g(z_0) = \frac{f^{(n)}(z_0)}{n!} \neq 0\]

\item[(2) $\Rightarrow$ (1)] Assume $f(z) = (z - z_0)^ng(z)$ for some holomorphic function $g$ such that $\phi(z_0) \neq 0$. Since $g$ is holomorphic at $z_0$, then there's a disk $D_\epsilon(z_0)$ on which $g$ has a Taylor expansion. Hence,
\begin{align*}
f(z) &= (z - z_0)^ng(z)\\[0.5em]
 &= (z - z_0)^n\sum_{k=0}^\infty \frac{g^{(k)}(z_0)}{k!}(z - z_0)^k\\[0.5em]
 &= \sum_{k=0}^\infty \frac{g^{(k)}(z_0)}{k!}(z - z_0)^{k + n}\\[0.5em]
 &= \sum_{k=n}^\infty \frac{g^{(k - n)}(z_0)}{(k - n)!}(z - z_0)^k
\end{align*}
Since Taylor expansions are unique (Theorem \ref{taylorunique}), the above series is the Taylor expansion of $f$ on $D_\epsilon(z_0)$, so
\begin{align*}
\frac{f^{(k)}(z_0)}{k!} &= 0,\quad 0 \leq k < n\\[0.5em]
\frac{f^{(n)}(z_0)}{n!} &= g(z_0) \neq 0
\end{align*}
Therefore $f^{(n)}(z_0) \neq 0$ and $f^{(k)}(z_0) = 0$ for $0 \leq k < n$. Hence $z_0$ is a zero of $f$ of order $n$.
\end{itemize}
\vspace*{-\baselineskip}
\end{proof}

\vspace*{1em}

\begin{example}
The function $p(z) = z^3 - 1$ has a zero at $1$. Define $g(z) = 1 + z + z^2$, then
\[p(z) = (z - 1)(1 + z + z^2) = (z - 1)g(z)\]
Clearly $g(z)$ is holomorphic at $1$ and $g(1) = 3 \neq 0$. So, by Theorem \ref{zero-order}, $p$ has a simple zero at $1$.
\end{example}

\vspace*{1em}

\begin{theorem}[Zeros of non-zero Holomorphic Functions]
Suppose that
\begin{itemize}
\item[(a)] $f$ is holomorphic at $z_0$;
\item[(b)] $f(z_0) = 0$ but $f$ is not identically zero on any neighbourhood of $z_0$.
\end{itemize}
Then $z_0$ is an isolated zero of $f$.
\end{theorem}
\begin{proof}
By (a), there's a disk $D_R(z_0)$ on which we can write
\[f(z) = \sum_{k=0}^\infty\frac{f^{(k)}(z_0)}{k!}(z - z_0)^k\]
Consider the set $S = \{m \in \zz\ :\ f^{(k)}(z_0) = 0,\text{for }\ 0 \leq k < m\}$, it is non-empty since $1 \in S$ as $f(z_0) = 0$. Note that $S$ is either a singleton, or the set of all positive integers. The latter implies $f$ is identically zero, which cannot be the case by (b); hence we have the former. Thus $S = \set{n}$ for some $n > 0$, so we have \[f(z_0) = f'(z_0) = \cdots = f^{(n-1)}(z_0) = 0 \quad \text{and} \quad f^{(n)}(z_0) \neq 0.\] Thus, $f(z)$ has a zero of order $n$ at $z_0$, and so
\[f(z) = (z - z_0)^ng(z)\]
for a holomorphic function $g(z)$ such that $g(z_0) \neq 0$. Since $g$ is continuous and non-zero at $z_0$, there exists a disk $D_\epsilon(z_0)$ on which $g(z) \neq 0$ for every $z \in D_\epsilon(z_0)$. Therefore, $f(z) \neq 0$ for every $z \in D_\epsilon(z_0)\setminus\set{z_0}$, and hence $z_0$ is an isolated zero of $f$.
\end{proof}

\vspace*{2em}

\begin{mdframed}
\begin{center}
{\Large Zeros and Poles}
\end{center}
\end{mdframed}
\begin{theorem}[Zeros and Poles]\label{zero-pole}
Suppose that
\begin{itemize}
\item[(a)] $p(z)$ and $q(z)$ are holomorphic at $z_0 \in \cc$; and
\item[(b)] $p(z_0) \neq 0$ and $q(z)$ has a zero of order $n$ at $z_0$,
\end{itemize}
then $f(z) = \dfrac{p(z)}{q(z)}$ has a pole of order $n$ at $z_0$.
\end{theorem}
\begin{proof}
Since $q(z)$ is holomorphic at $z_0$ and has a zero of order $n$ at $z_0$, by the (proof of) preceding theorem $z_0$ is an isolated zero. Hence $f$ has an isolated singularity at $z_0$.\\[0.5em]
Since $z_0$ is a zero of order $n$, then 
\[q(z) = (z - z_0)^ng(z)\]
for a holomorphic function $g(z)$ such that $g(z_0) \neq 0$. Writing $\phi(z) = \dfrac{p(z)}{g(z)}$, we have
\[f(z) = \frac{p(z)}{q(z)} = \frac{p(z)}{(z - z_0)^ng(z)} = \frac{\phi(z)}{(z - z_0)^n}\]
Moreover, $\phi(z_0) \neq 0$ and is holomorphic at $z_0$ since $p(z)$ and $g(z)$ are holomorphic at $z_0$ and $g(z_0) \neq 0$. The claim follows.
\end{proof}

\vspace*{1em}

\begin{example}
Consider 
\[f(z) = \frac{1}{1 - \cos z}\]
Using Theorem \ref{zero-pole}, we show that $f$ has a pole of order $2$ at $0$.\\
\\
Let $p(z) = 1$ and $q(z) = 1 - \cos z$, clearly they're both holomorphic at $0$. Moreover $p(0) = 1 \neq 0$ and $q(z)$ has a zero of order $2$ since
\begin{align*}
q(0) &= 1 - \cos 0 = 0\\[0.5em]
q'(0) &= \sin 0 = 0\\[0.5em]
q''(0) &= \cos 0 = 1 \neq 0
\end{align*}
Therefore, by Theorem \ref{zero-pole}, $f$ has a pole of order $2$ at $0$.
\end{example}

\vspace*{1em}

\begin{theorem}[Residue at a Simple Pole]\label{respolesimp}
Suppose that $p(z)$ and $q(z)$ are holomorphic at $z_0 \in \cc$. If $p(z_0) \neq 0$ and $z_0$ is a simple zero of $q(z)$, then 
\[\res{z_0}\frac{p(z)}{q(z)} = \frac{p(z_0)}{q'(z_0)}\]
\end{theorem}
\begin{proof}
By Theorem \ref{zero-pole}, $p(z)/q(z)$ has a simple pole at $z_0$. As in the proof of Theorem \ref{zero-pole}, we first write $q(z) = (z - z_0)g(z)$ and then
\[\frac{p(z)}{q(z)} = \frac{\phi(z)}{(z - z_0)}\]
where $\phi(z) = p(z)/g(z)$ where $g$ is a holomorphic function with $g(z_0) \neq 0$. Therfore, by Theorem \ref{poleorder},
\[\res{z_0}\frac{p(z)}{q(z)} = \phi(z_0) = \frac{p(z_0)}{g(z_0)}\]
Since $q(z) = (z - z_0)g(z)$, we have that $q'(z) = g(z) + (z - z_0)g'(z)$ and therefore $q'(z_0) = g(z_0)$. Hence
\[\res{z_0}\frac{p(z)}{q(z)} = \frac{p(z_0)}{q'(z_0)}\]
\end{proof}

\vspace*{1em}

\begin{example}\hfill
\begin{itemize}[itemsep=1em]
\item[(1)] Consider \[f(z) = \cot z = \frac{\cos z}{\sin z}\] Let $p(z) = \cos z$ and $q(z) = \sin z$, and $z_k = k\pi,\ k \in \zz$. Clearly both $p$ and $q$ are holomorphic at $z_k$ as they are entire. Moreover,
\begin{align*}
p(z_k) &= \cos k\pi = (-1)^k \neq 0\\[0.5em]
q(z_k) &= \sin k\pi = 0\\[0.5em]
q'(z_k) &= \cos k\pi = (-1)^k \neq 0
\end{align*}
Hence, $z_k$ is a simple pole for each $k \in \zz$, and
\[\res{z_k}\cot z = \frac{p(z_k)}{q'(z_k)} = \frac{(-1)^k}{(-1)^k} = 1\]
Let $C$ be the positively oriented circle of radius $k\pi + 1$ centered at $0$. Then by Cauchy's Residue theorem
\begin{align*}
\int_C\,\cot z\,dz &= 2\pi i\sum_{n=-k}^k\res{z_n}\cot z\\[0.5em]
 &= 2\pi i(2k + 1)
\end{align*}

\item[(2)] Consider \[f(z) = \frac{z - \sin z}{z^2\sin z}\] Let $p(z) = z - \sin z$ and $q(z) = z^2\sin z$. Clearly both $p$ and $q$ are holomorphic at $\pi$ as they are entire. Moreover,
\begin{align*}
p(\pi) &= \pi - \sin \pi = \pi \neq 0\\[0.5em]
q(\pi) &= \pi^2\sin \pi = 0\\[0.5em]
q'(\pi) &= 2\pi\sin \pi + \pi^2\cos \pi = -\pi^2 \neq 0
\end{align*}
Hence, $\pi$ is a simple pole and
\[\res{\pi}\frac{z - \sin z}{z^2\sin z} = \frac{p(\pi)}{q'(\pi)} = \frac{\pi}{-\pi^2} = -\frac{1}{\pi}\]

\item[(3)] Consider \[f(z) = \frac{z}{z^4 + 4}\] Let $p(z) = z$ and $q(z) = z^4 + 4$. Clearly both $p$ and $q$ are holomorphic at $1 + i$ as they are entire. Moreover,
\begin{align*}
p(1+i) &= 1 + i \neq 0\\[0.5em]
q(1+i) &= (1+i)^4 + 4 = 0\\[0.5em]
q'(1+i) &= 4(1+i)^3 \neq 0
\end{align*}
Hence, $1+i$ is a simple pole and
\[\res{1+i}\frac{z}{z^4 + 4} = \frac{p(1+i)}{q'(1+i)} = \frac{1+i}{4(1+i)^4} = \frac{1}{8i}\]
\end{itemize}
\end{example}

\vspace*{2em}

\subsection{Problems}
\vspace{0.1in}
To be added
%\begin{problem}\label{prob 12.1}
%
%\end{problem}