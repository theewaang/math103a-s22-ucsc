\vspace*{1em}

\begin{example}\hfill
\begin{itemize}[itemsep=1em]
\item[(1)] We've seen earlier that the function
\[g(z) = \begin{cases}\dfrac{\sin z}{z} & \text{if } z \neq 0\\[1em] 1 & \text{if } z = 0 \end{cases}\] 
is entire. Consider now the function
\[f(z) = \frac{\sin z}{z}\]
we have that $f$ has an isolated singularity at $0$. We have seen that its Laurent series on the annulus $0 < \abs{z} < \infty$ is
\[f(z) = \sum_{k=0}^\infty\frac{(-1)^kz^{2k}}{(2k + 1)!},\]
so the principal part of $f$ at $0$ is zero and thus $f$ has a removable singularity at $0$.

\item[(2)] The function
\[f(z) = \frac{1 - \cos z}{z^2}\]
also has a removable singularity at $0$. The Laurent series on $0 < \abs{z} < \infty$ is
\begin{align*}
\frac{1 - \cos z}{z^2} &= \frac{1}{z^2}\left(1 - \sum_{k=0}^\infty \frac{(-1)^k z^{2k}}{(2k)!}\right)\\[0.5em]
 &= -\frac{1}{z^2}\sum_{k=1}^\infty \frac{(-1)^k z^{2k}}{(2k)!}\\[0.5em]
 &= \sum_{k=1}^\infty \frac{(-1)^{k+1} z^{2k - 2}}{(2k)!}\\[0.5em]
 &= \sum_{k=0}^\infty \frac{(-1)^{k} z^{2k}}{(2k+2)!}
\end{align*}
So the principal part of $f$ at $0$ is zero, and therefore the function has a removable singularity at $0$. The entire function we obtain by ``removing the singularity" is
\[g(z) = \begin{cases}\dfrac{1 - \cos z}{z^2} & \text{if } z \neq 0\\[1em] \quad\dfrac{1}{2} & \text{if } z = 0 \end{cases}\]

\item[(3)] The function $f(z) = e^{1/z}$ has an isolated singularity at $0$. The Laurent series on $0 < \abs{z} < \infty$ is
\begin{align*}
e^{1/z} &= \sum_{k=0}^\infty\,\frac{1}{k!}\left(\frac{1}{z}\right)^k
\end{align*}
So the principal part of $f$ at $0$ has infinitely many terms, and therefore the function has an essential singularity at $0$.

\item[(4)] The function
\[f(z) = \frac{1}{z^2(1 - z)}\]
has an isolated singularity at $0$. The Laurent series on $0 < \abs{z} < 1$ is
\begin{align*}
\frac{1}{z^2(1 - z)} = \frac{1}{z^2}\cdot \frac{1}{1 - z} &= \frac{1}{z^2}\sum_{k=0}^\infty z^k\\[0.5em]
 &= \sum_{k=0}^\infty z^{k-2}\\[0.5em]
 &= \underbrace{\frac{1}{z^2} + \frac{1}{z}}_{\text{principal part}} + \sum_{k=0}^\infty z^k
\end{align*}
Clearly then $f$ has a pole of order $2$ at $0$.

\item[(5)] The function
\[f(z) = \frac{z^2 + z - 2}{z + 1}\]
has an isolated singularity at $-1$. The Laurent series on $0 < \abs{z + 1} < 1$ is
\begin{align*}
\frac{z^2 + z - 2}{z + 1} &= \frac{z(z + 1) - 2}{z + 1}\\[0.5em]
 &= z - \frac{2}{z + 1}\\[0.5em]
 &= -\frac{2}{z + 1} - 1 + (z + 1)
\end{align*}
Clearly then $f$ has a simple pole at $-1$.
\end{itemize}
\end{example}

\vspace*{2em}

\begin{mdframed}
\begin{center}
{\Large Residues at Poles}
\end{center}
\end{mdframed}
The following theorem gives a characterisation of poles and provides an efficient method for computing the residue at poles.
\begin{theorem}[Residue at Poles]\label{poleorder}
Let $z_0 \in \cc$ be an isolated singularity of $f$. Then the following are equivalent:
\begin{itemize}
\item[(1)] $z_0$ is a pole of order $n$.
\item[(2)] $f(z) = \dfrac{\phi(z)}{(z - z_0)^n}$ for some unique holomorphic function $\phi$ such that $\phi(z_0) \neq 0$.
\end{itemize}
Moreover, if (1) (or (2)) is true, the residue of $f$ at $z_0$ is given as
\[\res{z_0}f(z) = \frac{\phi^{(n-1)}(z_0)}{(n-1)!}\]
\end{theorem}
\begin{proof}\hfill
\begin{itemize}[leftmargin=4.5em,itemsep=1.5em]
\item[(1) $\Rightarrow$ (2)] Let's assume $z_0$ is a pole of order $n \geq 1$. Then
\[f(z) = \sum_{k=0}^\infty\,a_k(z - z_0)^k + \frac{a_{-1}}{z - z_0} + \frac{a_{-2}}{(z - z_0)^2} + \cdots + \frac{a_{-n}}{(z - z_0)^n}\]
with $a_{-n} \neq 0$, on an annulus $0 < \abs{z - z_0} < R$. Define
\[\phi(z) = \begin{cases} (z - z_0)^nf(z) & \text{if } z \neq z_0\\[0.5em] \quad a_{-n} & \text{if } z = z_0 \end{cases}\]
Clearly $f(z) = \dfrac{\phi(z)}{(z - z_0)^n}$, and $\phi$ has the following series expansion on $\abs{z - z_0} < R$
\[\phi(z) = \sum_{k=0}^\infty\,a_k(z - z_0)^{k + n} + a_{-1}(z - z_0)^{n-1} + a_{-2}(z - z_0)^{n-2} + \cdots + a_{-n}\]
Hence $\phi(z)$ is holomorphic on the disk and thus at $z_0$, and is necessarily unique. Moreover, $\phi(z_0) = a_{-n} \neq 0$.

\item[(2) $\Rightarrow$ (1)] Assume
\[f(z) = \frac{\phi(z)}{(z - z_0)^n}\]
for some holomorphic function $\phi$ such that $\phi(z_0) \neq 0$. Since $\phi$ is holomorphic at $z_0$, then there's a disk $\abs{z - z_0} < R$ on which $\phi$ has a Taylor expansion
\[\phi(z) = \sum_{k=0}^\infty \frac{\phi^{(k)}(z_0)}{k!}(z - z_0)^k\]
Hence,
\begin{align*}
f(z) &= \frac{\phi(z)}{(z - z_0)^n}\\[0.5em]
 &= \frac{1}{(z - z_0)^n}\sum_{k=0}^\infty \frac{\phi^{(k)}(z_0)}{k!}(z - z_0)^k\\[0.5em]
 &= \sum_{k=0}^\infty \frac{\phi^{(k)}(z_0)}{k!}(z - z_0)^{k - n}\\[0.5em]
 &= \frac{\phi(z_0)}{(z - z_0)^n} + \frac{\phi^{(1)}(z_0)}{1!(z - z_0)^{n - 1}} + \cdots + \frac{\phi^{(n-1)}(z_0)}{(n-1)!(z - z_0)} + \sum_{k=n}^\infty \frac{\phi^{(k)}(z_0)}{k!}(z - z_0)^{k - n}\\[0.5em]
 &= \frac{\phi(z_0)}{(z - z_0)^n} + \frac{\phi^{(1)}(z_0)}{1!(z - z_0)^{n - 1}} + \cdots + \frac{\phi^{(n-1)}(z_0)}{(n-1)!(z - z_0)} + \sum_{k=0}^\infty \frac{\phi^{(k + n)}(z_0)}{(k+n)!}(z - z_0)^{k}
\end{align*}
Since $\phi(z_0) \neq 0$ by assumption, this tells us that $f$ has a pole of order $n$ at $z_0$. Moreover, it's clear from the series that
\[\res{z_0}f(z) = \frac{\phi^{(n-1)}(z_0)}{(n-1)!}\]
\end{itemize}
\vspace*{-\baselineskip}
\end{proof}

\vspace*{1em}

\begin{example}\hfill
\begin{itemize}[itemsep=1em]
\item[(1)] The function $f(z) = \dfrac{z + 4}{z^2 + 1}$ has isolated singularities at $\pm i$.\\[0.5em]
Let's compute the residue first at $i$. Define
\[\phi(z) = \frac{z + 4}{z + i}\]
then clearly $f(z) = \dfrac{\phi(z)}{z - i}$ and $\phi$ is holomorphic and non-zero at $i$.\\[0.5em]
By Theorem \ref{poleorder}, $i$ is a simple pole and
\[\res{i}f(z) = \phi(i) = \frac{i + 4}{2i}\]
For $-i$, define \[\phi(z) = \frac{z + 4}{z - i}\]
then clearly $f(z) = \dfrac{\phi(z)}{z + i}$ and $\phi$ is holomorphic and non-zero at $-i$.\\[0.5em]
By Theorem \ref{poleorder}, $-i$ is a simple pole and
\[\res{-i}f(z) = \phi(i) = \frac{i - 4}{2i}\]

\item[(2)] The function $f(z) = \dfrac{z^3 + 2z}{(z - i)^3}$ has an isolated singularity at $i$.\\[0.5em]
Let's compute the residue at $i$. Define $\phi(z) = z^3 + 2z$. Then clearly
\[\phi(z) = \frac{\phi(z)}{(z - i)^3}\]
and $\phi$ is holomorphic and non-zero at $i$. By Theorem \ref{poleorder}, $i$ is a pole of order $3$ and
\[\res{i}f(z) = \frac{\phi''(i)}{2!} = \frac{6i}{2} = 3i\]

\item[(3)] The function $f(z) = \dfrac{(\log z)^3}{z^2 + 1}$ has an isolated singularity at $\pm i$. Here $\log z$ is the branch
\[\log z = \ln\abs{z} + i\arg z, \quad 0 < \arg z < 2\pi\]
Let's compute the residue at $i$. Define
\[\phi(z) = \frac{(\log z)^3}{z + i}\]
then clearly $f(z) = \dfrac{\phi(z)}{z - i}$ and $\phi$ is holomorphic and non-zero at $i$ since
\[\phi(i) = \frac{(\log i)^3}{2i} = \frac{(i\pi/2)^3}{2i} = -\frac{\pi^3}{16}\]
By Theorem \ref{poleorder}, $i$ is a simple pole and
\[\res{i}f(z) = \phi(i) = -\frac{\pi^3}{16}\]
\end{itemize}
\end{example}

\vspace*{2em}

\begin{mdframed}
\begin{center}
{\Large Zeros of Holomorphic Functions}
\end{center}
\end{mdframed}
\begin{definition}[Zeros of Holomorphic Functions]
Suppose that $f$ is holomorphic at $z_0 \in \cc$. We say that $f$ has a \cdef{zero\ of\ order} {\color{blue}$n$} \cdef{at} {\color{blue}$z_0$} if for some $n$ we have $f^{(n)}(z_0) \neq 0$ and $f^{(k)}(z_0) = 0$ for $0 \leq k < n$, in particular $f(z_0) = 0$. A zero of order $1$ is called a \cdef{simple\ zero}.\\[0.5em]
A zero is \cdef{isolated} if there exists an $\epsilon > 0$ such that $f(z) \neq 0$ for every $z \in D_\epsilon(z_0)\setminus\set{z_0}$.
\end{definition}

\vspace*{2em}

\subsection{Problems}
\vspace{0.1in}
To be added
%\begin{problem}\label{prob 12.1}
%
%\end{problem}