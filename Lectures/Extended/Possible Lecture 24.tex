\vspace*{1em}

\begin{theorem}[Residue at $\infty$]\label{resinfty}
If a function is holomorphic everywhere on $\cc$ except at a finite number of singularities lying interior to a simple closed positively oriented contour $C$, then
\[\int_C\,f(z)\,dz = 2\pi i\cdot\res{0}\left(\frac{1}{z^2}\,f\!\left(\frac{1}{z}\right)\right)\]
In particular,
\[\res{\infty}f(z) = -\res{0}\left(\frac{1}{z^2}\,f\!\left(\frac{1}{z}\right)\right)\]
\end{theorem}
\begin{proof}
Let $C_0$ be a negatively oriented circle centered at $0$, whose interior contains $C$.
\[\text{\color{red}add image here}\]
By Principle of Deformation of Paths (Corollary \ref{deformation}),
\begin{align*}
\int_C\,f(z)\,dz &= \int_{-C_0}\,f(z)\,dz\\[1em]
 &= -\int_{C_0}\,f(z)\,dz\\[0.5em]
 &= -\res{\infty}f(z)
\end{align*}
To compute $\res{\infty}f(z)$ in terms of $\res{0}f(z)$, we find the Laurent series of $f$ on an annulus $R < \abs{z} < \infty$ where 
\[\max_{w\in C}\abs{w} < R < \text{radius of $C_0$}\]
We have
\[f(z) = \sum_{k=0}^\infty\, a_kz^k + \sum_{k=1}^\infty\frac{a_{-k}}{z^k}\]
where
\[a_n = \frac{1}{2\pi i}\int_{-C_0}\frac{f(z)}{z^{n+1}}\,dz \quad \text{for } n \in \zz\]
Then,
\[\frac{1}{z^2}\,f\!\left(\frac{1}{z}\right) = \sum_{k=0}^\infty \frac{a_k}{z^{k+2}} + \sum_{k=1}^\infty\,a_{-k}z^{k-2},\quad \text{for }0 < \abs{\frac{1}{z}} < \frac{1}{R}\]
Note that,
\begin{align*}
\res{0}\left(\frac{1}{z^2}\,f\!\left(\frac{1}{z}\right)\right) = a_{-1} &= \frac{1}{2\pi i}\int_{-C_0}\,f(z)\,dz\\[1em]
&= -\res{\infty}f(z)
\end{align*}
\end{proof}

\vspace*{1em}

\begin{example}[Revisiting Example \ref{resexample}]
Let $C$ be the circle $\abs{z} = 2$ with positive orientation and $f(z) = \dfrac{4z - 5}{z(z - 1)}$, we can compute
\[\int_C\frac{4z - 5}{z(z - 1)}\,dz\]
using a single residue. $f(z)$ has isolated singularities at $0$ and $1$, both of which lie interior to $C$. We have
\begin{align*}
\frac{1}{z^2}\,f\!\left(\frac{1}{z}\right) &= \frac{1}{z^2}\left(\frac{4/z - 5}{1/z(1/z - 1)}\right)\\[0.5em]
 &= \frac{4 - 5z}{z(1-z)}
\end{align*}
Since $\dfrac{1}{1 - z}$ is holomorphic at $0$, it has a Maclaurin series in the unit disk around $0$, so
\begin{align*}
\frac{1}{z^2}\,f\!\left(\frac{1}{z}\right) = \frac{4 - 5z}{z(1-z)} &= \frac{4 - 5z}{z}\cdot\frac{1}{1 - z}\\[0.5em]
 &= \left(\frac{4}{z} - 5\right)\sum_{k=0}^\infty z^k\\[0.5em]
 &= 4\sum_{k=0}^\infty z^{k-1} - 5\sum_{k=0}^\infty z^k\\[0.5em]
 &= \frac{4}{z} - \sum_{k=0}^\infty z^k
\end{align*}
Therefore,
\[\res{0}\left(\frac{1}{z^2}\,f\!\left(\frac{1}{z}\right)\right) = 4 \quad \text{and} \quad \int_C\frac{4z - 5}{z(z - 1)}\,dz = 2\pi i\cdot\res{0}\left(\frac{1}{z^2}\,f\!\left(\frac{1}{z}\right)\right) = 8\pi i\]
using Theorem \ref{resinfty}.
\end{example}

\vspace*{1em}

\begin{example}
The function
\[f(z) = \frac{z^2}{(z - 2)(z^2 + 3)}\]
has no antiderivative on the domain $G = \setp{z \in \cc}{\abs{z} > 2}$.\\
\\
Let $C$ be the circle of radius $3$ centered at $0$. Notice that all three singularities of $f$, which are $2,\,\pm i\sqrt{3}$, lie in the interior of $C$. Hence, by Theorem \ref{resinfty},
\[\int_C\,f(z)\,dz = 2\pi i\cdot\res{0}\left(\frac{1}{z^2}\,f\!\left(\frac{1}{z}\right)\right)\]
We have,
\begin{align*}
\frac{1}{z^2}\,f\!\left(\frac{1}{z}\right) &= \frac{1}{z^2}\left(\frac{1/z^2}{(1/z - 2)(1/z^2 + 3)}\right)\\[0.5em]
 &= \frac{1}{z(1-2z)(1+3z^2)}
\end{align*}
The functions $\dfrac{1}{1 - 2z}$ and $\dfrac{1}{1 + 3z^2}$ are holomorphic at $0$,  and so have a Maclaurin series in the disk $D_{1/6}(0)$, here
\[\abs{2z} < \frac{1}{3} < 1 \quad \text{and} \quad \abs{3z^2} < \frac{1}{2} < 1\]
Therefore,
\begin{align*}
\frac{1}{1 - 2z} &= \sum_{k=0}^\infty 2^kz^k & \frac{1}{1 + 3z^2} &= \sum_{k=0}^\infty (-3)^kz^{2k}
\end{align*}
Hence the Maclaurin series of the product $\dfrac{1}{(1-2z)(1 + 3z^2)}$ has constant term $1$, and thus
\[\res{0}\left(\frac{1}{z^2}\,f\!\left(\frac{1}{z}\right)\right) = 1, \quad \text{therefore } \int_C\,f(z)\,dz = 2\pi i\cdot\res{0}\left(\frac{1}{z^2}\,f\!\left(\frac{1}{z}\right)\right) = 2\pi i \neq 0\]
Hence $f(z)$ doesn't have an antiderivative on $G$ by the Fundamental Theorem of Contour Integration.
\end{example}

\vspace*{2em}

\begin{mdframed}
\begin{center}
{\Large Classifying Isolated Singularities}
\end{center}
\end{mdframed}

\begin{discussion}
Recall that if $f$ has an isolated singularity at $z_0 \in \cc$, then $f$ has a Laurent series
\[f(z) = \sum_{k=0}^\infty\,a_k(z - z_0)^k + \sum_{k=1}^\infty \frac{a_{-k}}{(z - z_0)^k}\]
on some annulus $0 < \abs{z - z_0} < R$. The sum
\[\sum_{k=1}^\infty \frac{a_{-k}}{(z - z_0)^k} = \frac{a_{-1}}{z - z_0} + \frac{a_{-2}}{(z - z_0)^2} + \frac{a_{-3}}{(z - z_0)^3} + \cdots\]
is called the \cdef{principal\ part\ of} {\color{blue}$f$} \cdef{at} {\color{blue}$z_0$}.\\
\\
We will classify isolated singularities bases on what the principal part looks like: whether it's zero, non-zero with finitely many terms, or infinitely many terms. The aim is to understand how to compute residues based on the type of singularity.
\end{discussion}

\vspace*{1em}

\begin{definition}[Types of Singularities]
Suppose that $f$ has an isolated singularity at $z_0 \in \cc$ with principal part 
\[\sum_{k=1}^\infty \frac{a_{-k}}{(z - z_0)^k} = \frac{a_{-1}}{z - z_0} + \frac{a_{-2}}{(z - z_0)^2} + \frac{a_{-3}}{(z - z_0)^3} + \cdots\]
then
\begin{itemize}
\item[$\bullet$] $z_0$ is said to be a \cdef{removable\ singularity} if the principal part of $f$ at $z_0$ is zero, that is, $a_{-k} = 0$ for all $k \geq 1$.
\item[$\bullet$] $z_0$ is said to be a \cdef{essential\ singularity} if the principal part of $f$ at $z_0$ has infinitely many terms.
\item[$\bullet$] $z_0$ is said to be a \cdef{pole\ of\ order} {\color{blue}$n$} if for some $n$ we have $a_{-n} \neq 0$ and $a_{-k} = 0$ for every $k \geq n$. That is, the principal part has finitely many terms and is of the form
\[\frac{a_{-1}}{z - z_0} + \frac{a_{-2}}{(z - z_0)^2} + \cdots + \frac{a_{-n}}{(z - z_0)^n}\]
A pole of order $n = 1$ is called a \cdef{simple\ pole}.
\end{itemize}
\end{definition}

\vspace*{1em}

\begin{remark}[Removable Singularity]
Suppose that $f$ has a removable singularity at $z_0 \in \cc$, then by definition we can write
\[f(z) = \sum_{k=0}^\infty\, a_k(z - z_0)^k + 0\]
on some annulus $0 < \abs{z - z_0} < R$. Let's define
\[g(z) = \begin{cases} f(z) & \text{if } z \neq z_0\\[0.5em] a_0 & \text{if } z = z_0 \end{cases}\]
then
\[g(z) = \sum_{k=0}^\infty\, a_k(z - z_0)^k\]
on the disk $\abs{z - z_0} < R$. Therefore $g$ is holomorphic on the disk, and agrees with $f$ everywhere on the annulus $0 < \abs{z - z_0} < R$. In this way, the singularity has been removed and we have obtained a holomorphic function $g$ from $f$. This is the first example of \emph{analytic continuation}.
\end{remark}

\vspace*{2em}

\subsection{Problems}
\vspace{0.1in}
To be added
%\begin{problem}\label{prob 12.1}
%
%\end{problem}