\vspace*{1em}

\begin{mdframed}
\begin{center}
{\Large The Logarithmic Function}
\end{center}
\end{mdframed}

\begin{discussion}
The complex logarithmic function arises, just the like the usual real logarithmic function, from trying to solve the following equation for $w$
\[e^w = z\quad (z \neq 0)\]
Write $z = re^{i\theta}$ and $w = u + iv$, then 
\[e^u e^{iv} = e^w = z = re^{i\theta}.\]
So, $e^u = r$, giving us $u = \ln r = \ln\abs{z}$, and $v = \theta + 2k\pi$ for some $k \in \zz$, that is the possible values of $v$ are exactly $\arg z = \parg z + 2k\pi,\ k \in \zz$.\\[0.5em]
Therefore, 
\begin{align*}
w &= \ln\abs{z} + i\arg(z)\\[0.5em]
&= \ln\abs{z} + i\parg(z) + 2k\pi i,\ k \in \zz
\end{align*}
Essentially, $w$ is not unique, as $v$ is not unique. This is to be expected, since $e^z$ is not injective as it is periodic.\\
\\
Multiple functions satisfy the equation we considered, which we package into a \emph{multi-valued function} using $\arg z$.
\end{discussion}

\vspace*{1em}

\begin{definition}[The Logarithmic Function]
We define the \cdef{logarithmic\ function} $\log z$ for any $z \neq 0$, following the discussion above, as
\[\log z \coloneqq \ln\abs{z} + i\arg(z)\]
Note that $\log z$ is not really a function but a \emph{multi-valued function}, as $\arg z$ is not single-valued.\\
\\
The \cdef{principal\ logarithm}, denoted $\plog z$, is defined by taking the principal argument of $z$
\[\plog z \coloneqq \ln\abs{z} + i\parg z,\quad -\pi < \parg z \leq \pi\]
The principal branch of $\log$ is a single-valued function.
\end{definition}

\vspace*{1em}

\begin{proposition}[Properties of the Logarithm]\label{proplog}
Consider $z \in \cc$. 
\begin{itemize}
\item[(1)] $e^{\log z} = z$.
\item[(2)] $\log e^z = z + 2k\pi i,\ k \in \zz$.
\item[(3)] $\log z = \plog z + 2k\pi i,\ k \in \zz$.
\item[(4)] If $z = x \in \rr_{>0}$, then $\plog z = \ln x$.
\end{itemize}
\end{proposition}
\begin{proof}\hfill
\begin{itemize}
\item[(1)] Note that 
\begin{align*}
e^{\log z} &= e^{\ln\abs{z} + i\arg z}\\[0.5em]
&= e^{\ln\abs{z}}e^{i(\parg z + 2k\pi)},\ k \in \zz\\[0.5em]
&= e^{\ln\abs{z}}e^{i\parg z}e^{2k\pi i},\ k \in \zz\\[0.5em]
&= \abs{z}e^{i\parg z}\\[0.5em]
&= z
\end{align*}
\item[(2)] Note that 
\begin{align*}
\log e^z &= \ln\abs{e^z} + i\arg(e^z)\\[0.5em]
&= \ln e^{\Re z} + i(\Im z + 2k\pi),\ k \in \zz\\[0.5em]
&= \Re z + i\Im z + 2k\pi i,\ k \in \zz\\[0.5em]
&= z + 2k\pi i,\ k \in \zz
\end{align*}
\item[(3)] Note that
\begin{align*}
\log z &= \log e^{\plog z},\ \text{by (1)}\\[0.5em]
&= \plog z + 2k\pi i,\ k \in \zz,\ \text{by (2)}
\end{align*}
\item[(4)] Note that if $z = x \in \rr_{>0}$, then $\parg z = 0$, therefore
\[\plog z = \ln\abs{z} + i\parg z = \ln x.\]
\end{itemize}
\vspace*{-\baselineskip}
\end{proof}

\vspace*{1em}

\begin{example}\label{logcalc}\hfill
\begin{itemize}[itemsep=1.5em]
\item[(1)] $\begin{aligned}[t]
\log(1 + i\sqrt{3}) &= \ln\abs{1 + i\sqrt{3}} + i\arg(1 + i\sqrt{3})\\[0.5em]
&= \ln 2 + i\left(\dfrac{\pi}{3} + 2k\pi\right),\ k \in \zz\\[1em]
\plog(1 + i\sqrt{3}) &= \ln 2 + \dfrac{\pi i}{3}\\[1em]
\end{aligned}$
\end{itemize}

\begin{multicols}{2}
\begin{itemize}
\item[(2)] $\begin{aligned}[t]
\log 1 &= \ln\abs{1} + i\arg 1\\[0.5em]
&= 0 + i\left(0 + 2k\pi\right),\ k \in \zz\\[0.5em]
&= 2k\pi i,\ k \in \zz\\[1em]
\plog 1 &= 0\\[1em]
\end{aligned}$

\item[(3)] $\begin{aligned}[t]
\log -1 &= \ln\abs{-1} + i\arg 1\\[0.5em]
&= \ln 1 + i\left(\pi + 2k\pi\right),\ k \in \zz\\[0.5em]
&= (2k+1)\pi i,\ k \in \zz\\[1em]
\plog -1 &= \pi i\\[1em]
\end{aligned}$
\end{itemize}
\end{multicols}

\begin{itemize}
\item[(4)] Familiar properties of logarithms that you know may not hold.
\begin{itemize}
\item[(a)] $\plog(-1 + i)^2 \neq 2\plog (-1 + i)$
\begin{align*}
\plog(-1 + i)^2 = \plog(-2i) &= \ln\abs{-2i} + i\parg(-2i)\\[0.5em]
&= \ln 2 + i\left(-\dfrac{\pi}{2}\right)\\[0.5em]
&= \ln 2 - \dfrac{\pi i}{2}\\[1em]
2\plog (-1 + i) &= 2\ln\abs{-1 + i} + 2i\arg(-1 + i)\\[0.5em]
&= 2\ln\sqrt{2} + 2i\left(\dfrac{3\pi}{4}\right)\\[0.5em]
&= \ln 2 + \dfrac{3\pi i}{2}
\end{align*}
\item[(b)] $\log i^2 \neq 2\log i$
\begin{align*}
\log i^2 &= \log -1 = (2k+1)\pi i,\ k \in \zz\\[1em]
2\log i &= 2\ln\abs{i} + 2i\arg i\\[0.5em]
&= 0 + 2i\left(\dfrac{\pi}{2} + 2k\pi\right),\ k \in \zz\\[0.5em]
&= (4k + 1)\pi i,\ k \in \zz
\end{align*}
\end{itemize}
\end{itemize}
\end{example}

\vspace*{1em}

\begin{proposition}
For all $z,\, w\in \cc^*$
\begin{multicols}{2}
\begin{itemize}
\item[(1)] $\log zw = \log z + \log w$
\item[(2)] $\log w^{-1} = -\log w$
\end{itemize}
\end{multicols}
\emph{One treats this as an equality of sets. (1) and (2) also gives you $\log z/w = \log z - \log w$.}
\end{proposition}
\begin{proof}\hfill
\begin{itemize}
\item[(1)] We have
\begin{align*}
\log z + \log w &= \ln\abs{z} + i\arg z + \ln\abs{w} + i\arg w\\[0.5em]
&= \ln\abs{z}\abs{w} + i(\arg z + \arg w)\\[0.5em]
&= \ln\abs{zw} + i\arg zw,\ \text{by Proposition \ref{prodarg} (1)}\\[0.5em]
&= \log zw
\end{align*}
\item[(2)] We have
\begin{align*}
\log w^{-1} &= \ln|w^{-1}| + i\arg w^{-1}\\[0.5em]
&= \ln\abs{w}^{-1} + i(-\arg w),\ \text{by Proposition \ref{prodarg} (2)}\\[0.5em]
&= -(\ln\abs{w} + i\arg w)\\[0.5em]
&= -\log w
\end{align*}
\end{itemize}
This statement does not hold if we replace $\log z$ with $\plog z$.
\end{proof}

%\vspace*{1em}

\begin{definition}[Branch of a Multi-Valued Functions]
A \cdef{branch} of a multi-valued function $f$ is a single-valued function $F$ such that
\begin{itemize}
\item $F$ is holomorphic on some domain $G$; and
\item $F$ assigned to each $z\in G$ precisely one value $F(z)$ of $f(z)$.
\end{itemize}
%\[\text{\color{red}insert branch cut example}\]
A portion of a line or curve in the complex plane is called a \cdef{branch\ cut} for $f$ if a branch $f$ is defined on its complement. A point belonging to \emph{every} branch cut of $f$ is a \cdef{branch\ point}.
\end{definition}

\vspace*{1em}

\begin{proposition}[Branches of $\log$]\label{logbranch}
Let $\alpha \in \rr$. The function
\[L_{\alpha}(z) = L_{\alpha}(re^{i\theta}) = \ln r + i\theta,\quad \alpha < \theta < \alpha + 2\pi\]
is a branch of $f(z) = \log z$. Note that $\Re L_\alpha = u(r,\theta) = \ln r$ and $\Im L_\alpha = v(r,\theta) = \theta$.
\end{proposition}
\begin{proof}
We first remark that if we were to define $L_\alpha$ also on the ray $\theta = \alpha$, it would not be continuous there. For if $z$ is a point on that ray, as one notes that $\lim_{\theta \to \alpha^-}\theta = \alpha$ but $\lim_{\theta \to \alpha^+}\theta \neq \alpha$ as the points close to the ray to the right have arguments near $\alpha + 2\pi$.\\
\\
It is clear that $L_\alpha(z)$ is single-valued and, for each $z$, $L_\alpha(z)$ is a value of $\log z$. We need to show $L_\alpha$ is holomorphic. Note that $u(r,\theta) = \ln r$ and $v(r,\theta) = \theta$ have continuous partial derivatives on the domain of definition
\begin{align*}
u_r &= \frac{1}{r} & v_r &= 0\\[0.5em]
u_\theta &= 0 & v_\theta &= 1
\end{align*}
Clearly, the Polar Cauchy Riemann equations (\ref{pcreqex}) are satisfied, and therefore $L_\alpha$ is holomorphic. In fact,
\begin{align*}
L'_\alpha(z) &= e^{-i\theta} (u_r + iv_r) = e^{-i\theta}\left(\dfrac{1}{r}\right) = \frac{1}{z}
\end{align*}
In particular, $\plog z$ for those $z$ such that $-\pi < \parg z < \pi$ is a branch of $\log z$, called the \cdef{principal} \cdef{branch\ of\ the\ logarithm} and 
\[(\plog z)' = \frac{1}{z}\]
\end{proof}

\vspace*{1em}

\begin{remark}
The branch cut for $\log z$ in Proposition \ref{logbranch} is the ray $r> 0,\,\theta = \alpha$
\[\begin{tikzpicture}[scale=0.75]
    \draw[->,thick] (0,0)--(5,0);
	\draw[<->,thick] (0,-1)--(0,5);
	\draw[<-,>=stealth,thick,dashed,firebrick] (-6,0)--(0,0);

    \draw[->,>=stealth,thick,dashed,indigo] (0,0) -- (5,3);
    \node (a) at (2.5,1.5) {};
    \node (b) at (0,0) {};
    \node (c) at (0.2,0) {};
    \draw pic["$\color{indigo}\alpha$", ->,>=stealth,thick, draw=indigo, angle eccentricity=1.3, angle radius=1cm] {angle=c--b--a};

	\fill[teal] (0,0) circle (3pt);
    \node[] (3) at (-2.5,0.5) {\footnotesize\color{firebrick}branch cut for $\plog z$};
    \node[text width=2.8cm] (3) at (6.5,1.5) {\footnotesize\color{indigo}branch cut for $L_\alpha$ in Proposition \ref{logbranch}};

    \end{tikzpicture}\]
The branch cut for $\plog z$ is the ray $r> 0,\, \theta = \pi$, i.e., the negative real axis. The origin is a branch point of $\log z$.
\end{remark}

%\vspace*{1em}

\begin{example}[Integer Powers and Roots]
The logarithmic function can be used to compute integer powers and roots (as previously seen and defined).
\begin{itemize}
\item[(1)] $z^n = e^{n\log z}$
\item[(2)] $z^{1/n} = e^{\frac{\log z}{n}}$
\end{itemize}
\begin{proof}
We note that
\begin{align*}
e^{n\log z} &= e^{n(\ln\abs{z} + i\arg z)} & e^{\frac{\log z}{n}} &= e^{\frac{1}{n}\left(\ln\abs{z} + i\arg z\right)} \\[0.5em]
 &= e^{n\ln\abs{z}}\cdot e^{in\arg z} & &= e^{\frac{1}{n}\ln\abs{z}}\cdot e^{i\left(\frac{\arg z}{n}\right)}\\[0.5em]
 &= \abs{z}^n\cdot (e^{i\arg z})^n & &= e^{\frac{1}{n}\ln\abs{z}}\cdot e^{i\left(\frac{\parg z + 2k\pi}{n}\right)}\\[0.5em]
 &= (\abs{z}e^{i\arg z})^n & &= \sqrt[n]{\abs{z}}\cdot e^{i\left(\frac{\parg z + 2k\pi}{n}\right)}\\[0.5em]
 &= z^n & &= z^{1/n}
\end{align*}
Recall that $z^n$ is single-valued, but $z^{1/n}$ is multi-valued, as complex numbers have $n$ distinct $n^{\text{th}}$ roots (Proposition \ref{distroot}). In fact, using the the principal logarithm, the complex number
\[e^{\frac{\plog z}{n}}\]
gives the principal $n^{\text{th}}$ root of $z$. 
\end{proof}
\end{example}

\vspace*{2em}

\begin{mdframed}
\begin{center}
{\Large Power and Exponential Functions}
\end{center}
\end{mdframed}

\begin{definition}[Power Function]
The \cdef{power\ function} $z^\alpha$ for a fixed $c \in \cc$ is the \emph{multi-valued} function
\[z^c \coloneqq e^{\,c\log z},\quad z \neq 0\]
\end{definition}

\vspace*{1em}

\begin{proposition}[Branches of $z^c$]
A branch of $z^c$ is determined by specifying a branch of $\log z$
\[\log z = \ln\abs{z} + i\arg z,\quad z \neq 0,\ \alpha < \arg z < \alpha + 2\pi\]
Moreover,
\[(z^c)' = c z^{c-1},\]
whenever $z \neq 0,\ \alpha < \arg z < \alpha + 2\pi$.
\end{proposition}
\begin{proof}
We only need to verify that $z^c$ is holomorphic, once a branch of $\log z$ has been specified. Since $z^c = e^{\,c\log z}$ is a composition of two holomorphic functions, $z^c$ itself is holomorphic. Moreover, by the chain rule
\begin{align*}
(z^c)' = (e^{\,c\log z})' &= e^{\,c\log z}(c\log z)'\\[0.5em]
&= e^{\,c\log z}\cdot \frac{c}{z}\\[0.5em]
&= c\cdot \frac{e^{\,c\log z}}{e^{\log z}} = c\cdot e^{(c-1)\log z} = cz^{c-1}
\end{align*}
\end{proof}

%\vspace*{1em}

\begin{discussion}
The \cdef{principal\ branch} of $z^c$ is defined by specifying the principal branch $\plog z$ of $\log z$. The principal branch of $z^c$ reduces to the usual power function when $z =  x \in \rr$.
\end{discussion}

\vspace*{2em}

\subsection{Problems}
\vspace{0.1in}

\begin{problem}\label{prob 9.1}
Find the all possible values of
\begin{multicols}{2}
\begin{itemize}
\item[(a)] $\log(-5)$
\item[(b)] $\log(-2 + 2i)$
\item[(c)] $\log(\sqrt{2} + i\sqrt{6})$
\item[(d)] $\log(-ei)$
\item[(e)] $\log(1 + i)$
\item[(f)] $\log(-\sqrt{3} + i)$
\end{itemize}
\end{multicols}
\end{problem}

\vspace{0.1in}

\begin{problem}\label{prob 9.2}
Compute
\begin{multicols}{2}
\begin{itemize}
\item[(a)] $\plog(6-6i)$
\item[(b)] $\plog(-e^2)$
\item[(c)] $\plog(-12 + 5i)$
\item[(d)] $\plog((1 + i\sqrt{3})^5)$
\item[(e)] $\plog(3 - 4i)$
\item[(f)] $\plog((1+i)^4)$
\end{itemize}
\end{multicols}
\end{problem}

\vspace{0.1in}

\begin{problem}\label{prob 9.3}\hfill
\begin{itemize}
\item[(a)] Show that if $\Re z_1 > 0$ and $\Re z_2 > 0$, then
\[\plog(z_1z_2) = \plog z_1 + \plog z_2.\]
\item[(b)] Show that for any two non-zero complex numbers $z_1$ and $z_2$,
\[\plog(z_1z_2) = \plog z_1 + \plog z_2 + 2N\pi i,\]
where $N \in \set{0,\pm 1}$.
\end{itemize}
\end{problem}

\vspace{0.1in}

\begin{problem}\label{prob 9.4}
Example \ref{logcalc} (4) tells us that it's not necessarily true that $\log z^n = n\log z$, for $n \in \zz_{>0}$.\\[0.5em]
Writing $z = re^{i\parg z}$, show that, where $n \in \zz_{>0}$
\[\log(z^{1/n}) = \frac{1}{n}\ln r + i\left(\frac{\parg z + 2(pn + k)\pi}{n}\right),\quad k = 0,\ldots,n-1.\]
Now, after writing 
\[\frac{1}{n}\log z = \frac{1}{n}\ln r + i\left(\frac{\parg z + 2qz}{n}\right),\quad q \in \zz,\]
show that we have equality of sets
\[\log(z^{1/n}) = \frac{1}{n}\log z\]
\end{problem}

\vspace{0.1in}

\begin{problem}\label{prob 9.5}
Find a domain in which the given function $f$ is holomorphic; then find the derivative $f'$.
\begin{itemize}
\item[(a)] $f(z) = 3z^2 - e^{2iz} + i\plog z$
\item[(b)] $f(z) = (z + 1)\plog z$
\item[(c)] $f(z) = \dfrac{\plog(2z-i)}{z^2 + 1}$
\item[(d)] $f(z) = \plog(z^2 + 1)$
\end{itemize}
\end{problem}