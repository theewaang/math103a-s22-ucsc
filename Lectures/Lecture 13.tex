\vspace*{1em}

Some examples involving a branch of a multi-valued function. 

\begin{example}\hfill
\begin{itemize}[itemsep=1.5em]
\item[(4)] Integrate the branch of square root \[f(z) = z^{1/2} = e^{(1/2)\log z},\quad \abs{z}>0,\ 0 < \arg z < 2\pi\] over the contour
\[\begin{tikzpicture}[scale=1.4]
    \draw[<-,thick] (-1.75,0)--(0,0);
	\draw[<->,thick] (0,-0.5)--(0,1.75);
    \begin{scope}
        \node[label=right:{$C$}](A) at (1,1) {};
        \node[label=below:{$-R$}](B) at (-1.25,0) {};
        \node[label=below:{$R$}](C) at (1.25,0) {};
        \draw[use Hobby shortcut,clockwise arrows,thick]
	(-1.25,0) .. (0,1.25) .. (1.25,0);
    \end{scope}
	\draw[->,thick,dashed,firebrick] (0,0)--(1.75,0);
    \draw [firebrick,fill=white] (0,0) circle (1.3pt);
    \draw [fill=black] (B) circle (1.3pt);
    \draw [fill=black] (C) circle (1.3pt);
\end{tikzpicture}\]
\[C:\ z(t) = Re^{it},\quad R>0,\ 0 \leq t \leq \pi\]
Note that $f(z)$ is not defined at the initial point $z = R$ of the contour $C$ as $\arg R = 0$, that is, $f(z(t))$ is not defined for $t = 0$. The integral
\[\int_C\,f(z)\ dz = \int_0^{\pi}\,f(z(t))\,z'(t)\ dt\]
nevertheless exists as the integrand $f(z(t))\,z'(t)$ is piecewise continuous on $[0,\pi]$. To see this, we note that for $0 < t \leq \pi$
\allowdisplaybreaks
\begin{align*}
f(z(t))\,z'(t) = e^{(1/2)\log Re^{it}}Rie^{it} &= iRe^{(\ln R + it)/2}e^{it}\\[0.5em]
 &= iR(R^{1/2}e^{it/2})e^{it}\\[0.5em]
 &= iR^{3/2}e^{3it/2}\\[0.5em]
 &= iR^{3/2}\left(\cos\frac{3t}{2} + i\sin\frac{3t}{2}\right) = R^{3/2}\left(-\sin\frac{3t}{2} + i\cos\frac{3t}{2}\right)
\end{align*}
The right hand limits of the real and imaginary parts of $f(z(t))\,z'(t)$ at $t = 0$ exist, and equal $0$ and $R^{3/2}$. Therefore, $f(z(t))\,z'(t)$ is continuous on $[0,\pi]$ with its value at $t = 0$ defined as $iR^{3/2}$. Hence, 
\begin{align*}
\int_C\,f(z)\ dz = \int_0^{\pi}\,f(z(t))\,z'(t)\ dt &= \int_0^{\pi}\,iR^{3/2}e^{3it/2}\ dt\\[0.5em]
&= iR^{3/2}\int_0^{\pi}\,e^{3it/2}\ dt\\[0.5em]
&= iR^{3/2}\Bigg[\frac{e^{3it/2}}{3i/2}\Bigg]_0^{\pi}\\[0.5em]
&= \frac{2}{3}R^{3/2}\left(e^{3\pi i/2} - e^0\right) = \frac{2}{3}R^{3/2}\left(-i - 1\right) = -\frac{2}{3}R^{3/2}\left(1 + i\right)
\end{align*}

\item[(5)] Integrate the principal branch of \[f(z) = z^{i-1} = e^{(i - 1)\plog z},\quad \abs{z}>0,\ -\pi < \parg z < \pi\] over the contour
\[\begin{tikzpicture}[scale=1.4]
    \draw[<-,thick] (1.75,0)--(0,0);
	\draw[<->,thick] (0,1.75)--(0,-1.75);
    \begin{scope}
        \node[label=right:{$C$}](A) at (1,1) {};
        \draw[use Hobby shortcut,clockwise arrowsend,thick]
	(-1.234,0.2) .. (0,1.25) .. (1.25,0) .. (0,-1.25) .. (-1.25,0);
    \end{scope}
	\draw[->,thick,dashed,firebrick] (0,0)--(-1.75,0);
    \draw [fill=black] (-1.25,0) circle (1.3pt);
    \draw [firebrick,fill=white] (0,0) circle (1.3pt);
\end{tikzpicture}\]
\[C:\ z(t) = e^{it},\quad R>0,\ -\pi \leq t \leq \pi\]
Since the curve crosses the branc cut, we need to check if integrand $f(z(t))\,z'(t)$ is piecewise continuous on $[-\pi,\pi]$. To see this, we note that for $-\pi < t \leq \pi$
\begin{align*}
f(z(t))\,z'(t) = e^{(i - 1)\plog e^{it}}ie^{it} &= ie^{(i-1)(\ln 1 + it)}e^{it}\\[0.5em]
 &= ie^{(i-1)it}e^{it}\\[0.5em]
 &= ie^{(i - 1)it + it} = ie^{i^2t} = ie^{-t}
\end{align*}
The right hand limits of the real and imaginary parts of $f(z(t))\,z'(t)$ at $t = \pi$ exist, and equal $0$ and $e^{-\pi}$. Therefore, $f(z(t))\,z'(t)$ is continuous on $[-\pi,\pi]$ with its value at $t = -\pi$ defined as $ie^{-\pi}$. Hence, 
\begin{align*}
\int_C\,f(z)\ dz = \int_{-\pi}^{\pi}\,f(z(t))\,z'(t)\ dt &= \int_{-\pi}^{\pi}\,ie^{-t}\ dt\\[0.5em]
&= i\int_{-\pi}^{\pi}\,e^{-t}\ dt\\[0.5em]
&= i\Big[-e^{-t}\Big]_{-\pi}^{\pi}\\[0.5em]
&= i\left(-e^{-\pi} - (-e^{-(-\pi)})\right)\\[0.5em]
&= i\left(e^{\pi}-e^{-\pi}\right)
\end{align*}
\end{itemize}
\end{example}

\vspace*{2em}

\begin{mdframed}
\begin{center}
{\Large Estimating Contour Integrals}
\end{center}
\end{mdframed}

\begin{lemma}[Triangle Inequality for Integrals]\label{intriineq}
Suppose $\gamma:[a,b] \to \cc$ is piecewise continuous. Then
\[\abs{\int_a^b\,\gamma(t)\ dt} \leq \int_a^b\abs{\gamma(t)}dt\]
\end{lemma}
\begin{proof}
Let's first assume 
\[\int_a^b\,\gamma(t)\ dt = 0,\]
then the lemma holds as $\abs{\gamma(t)} \geq 0$ for all $t \in [a,b]$ and so its integral is non-negative. Otherwise, let \[r_0e^{it_0} = \int_a^b\,\gamma(t)\ dt \neq 0.\] Then,
\begin{align*}
\abs{\int_a^b\,\gamma(t)\ dt} = |r_0e^{it_0}| = r_0 = \Re r_0 &= \Re (r_0e^{it_0}e^{-it_0})\\[0.5em]
&= \Re \left(e^{-it_0}\int_a^b\,\gamma(t)\ dt\right)\\[0.5em]
&= \Re \left(\int_a^b\,e^{-it_0}\gamma(t)\ dt\right)\\[0.5em]
&= \int_a^b\,\Re (e^{-it_0}\gamma(t))\ dt\\[0.5em]
&\leq \int_a^b\,|e^{-it_0}\gamma(t)|\ dt,\ \text{using Discussion \ref{cmplxnorm}}\\[0.5em]
&= \int_a^b\,|e^{-it_0}|\abs{\gamma(t)}\ dt\\[0.5em]
&= \int_a^b\abs{\gamma(t)} dt\\[-3em]
\end{align*}
\end{proof}

\vspace*{1em}

\begin{theorem}[Bound for Contour Integrals]\label{contourtriineq}
Suppose that $C$ is a contour of length $L$ and $f$ is piecewise continuous on $C$. Then
\[\abs{\int_C\,f(z)\ dz} \leq \max_{z \in C}\abs{f(z)} \cdot L(C)\]
\end{theorem}
\begin{proof}
Suppose $z:[a,b] \to \cc$ parametrises $C$. By assumption $f(z(t))$ is piecewise continuous on $[a,b]$. Hence, $\max_{z \in C}\abs{f(z)} = \max_{t \in [a,b]}\abs{f(z(t))}$ is finite as $f(z(t))$ is continuous on a closed and bounded interval. Thus,
\begin{align*}
\abs{\int_C\,f(z)\ dz} &= \abs{\int_a^b\,f(z(t))\,z'(t)\ dz}\\[0.5em]
 &\leq \int_a^b\abs{f(z(t))\,z'(t)}\, dz,\ \text{by Lemma \ref{intriineq}}\\[0.5em]
 &= \int_a^b\abs{f(z(t))}\abs{z'(t)}\, dz\\[0.5em]
 &\leq \int_a^b\max_{t \in [a,b]}\abs{f(z(t))}\abs{z'(t)}\, dz\\[0.5em]
 &= \max_{t \in [a,b]}\abs{f(z(t))}\int_a^b\abs{z'(t)}\, dz = \max_{t \in [a,b]}\abs{f(z(t))}\cdot L(C) = \max_{z \in C}\abs{f(z)}\cdot L(C)\\[-3em]
\end{align*}
\end{proof}

%\vspace*{1em}

\begin{example}\hfill
\begin{itemize}
\item[(1)] Finding a bound for
\[\int_C\frac{z^2 + 1}{z^3 + 2}\ dz,\]
where $C$ is the semicircle $z(t) = 2e^{it},\ 0 \leq t \leq \pi$.\\
\\
All we need to find is an $M > 0$ such that, for all $z \in C$
\[\abs{\frac{z^2 + 1}{z^2 + 3}} \leq M,\qquad \text{because then}\quad\max_{z \in C}\abs{\frac{z^2 + 1}{z^2 + 3}} \leq M\]
Suppose $z \in C$, then $\abs{z} = 2$, and therefore
\[|z^2 + 1| \leq \abs{z}^2 + 1 = 5;\]
also,
\[|z^3 + 2| \geq |\abs{z}^3 - 2| = |2^3 - 2| = 6.\]
Together, we get, for any $z \in C$
\[\abs{\frac{z^2 + 1}{z^2 + 3}} \leq \frac{5}{6}.\]
Hence, 
\[\abs{\int_C\frac{z^2 + 1}{z^3 + 2}\ dz} \leq \max_{z \in C}\abs{\frac{z^2 + 1}{z^2 + 3}}\cdot L(C) \leq \frac{5}{6}\cdot L(C) = \frac{5}{6}\cdot 2\pi = \frac{5\pi}{3}\]

\item[(2)] Show that
\[\lim_{R \to \infty}\int_{C_R}\frac{z^2 + z}{z^4 + 2z^2 + 1}\ dz = 0,\]
where $C_R$ is the semicircle $z(t) = Re^{it},\ 0 \leq t \leq 2\pi$. Note that $L(C) = 2\pi R$.\\
\\
Let $z \in C_R$, then $\abs{z} = R$, and therefore
\[|z^2 + z| \leq \abs{z}^2 + z = R^2 + R;\]
also,
\[|z^4 + 2z^2 + 1| \geq |(z^2 + 1)| = |z^2 + 1|^2 \geq ||z|^2 - 1|^2 = |R^2 - 1|^2 = (R^2 - 1)^2.\]
Together, we get, for any $z \in C$ and $R > 1$
\[\abs{\frac{z^2 + z}{z^4 + 2z^2 + 1}} \leq \frac{R^2 + R}{(R^2 - 1)^2}.\]
Hence, 
\[\abs{\int_{C_R}\frac{z^2 + z}{z^4 + 2z^2 + 1}\ dz} \leq \frac{R^2 + R}{(R^2 - 1)^2}\cdot 2\pi R  \to 0,\ \text{as } R \to \infty\]
Therefore,
\[\lim_{R \to \infty}\int_{C_R}\frac{z^2 + z}{z^4 + 2z^2 + 1}\ dz = 0,\]
by the Sandwich theorem.
\end{itemize}
\end{example}

%\vspace*{1em}

\begin{example}[in-class]
Finding a bound for
\[\int_C\frac{z^2 - 1}{z^4 + 2}\ dz,\]
where $C$ is the sector $z(t) = 5e^{it},\ \pi/4 \leq t \leq 3\pi/4$.
\end{example}
\begin{proof}[Answer]
Let's first compute $L(C)$. We first note that $z'(t) = i5e^{it}$, therefore,
\begin{align*}
L(C) &= \int_{\pi/4}^{3\pi/4}\,\abs{z'(t)}\ dt\\[1em] 
 &= \int_{\pi/4}^{3\pi/4}\,\abs{5ie^{it}}\ dt\\[1em]
 &= \int_{\pi/4}^{3\pi/4}\,5\ dt\\[1em]
 &= 5\int_{\pi/4}^{3\pi/4}\,dt\\[1em]
 &= 5\left(\frac{3\pi}{4} - \frac{\pi}{4}\right)\\[1em]
 &= \frac{5\pi}{2}
\end{align*}
Now, suppose $z \in C$, then $\abs{z} = 5$, and therefore
\[|z^2 - 1| \leq \abs{z^2} + \abs{-1} = \abs{z}^2 + 1 = 26;\]
also,
\[|z^4 + 2| \geq |\abs{z^4} - \abs{2}| = |\abs{z}^4 - 2| = 623.\]
Together, we get, for any $z \in C$
\[\abs{\frac{z^2 - 1}{z^4 + 2}} \leq \frac{26}{623},\quad \text{hence }\max_{z \in C}\abs{\frac{z^2 - 1}{z^4 + 2}} \leq \frac{26}{623}\]
Hence, 
\[\abs{\int_C\frac{z^2 - 1}{z^4 + 2}\ dz} \leq \max_{z \in C}\abs{\frac{z^2 - 1}{z^4 + 2}}\cdot L(C) \leq \frac{26}{623}\cdot L(C) = \frac{26}{623}\cdot \frac{5\pi}{2} = \frac{65\pi}{623}\]
\end{proof}

\vspace*{2em}

\subsection{Problems}
\vspace{0.1in}
To be added
%\begin{problem}\label{prob 12.1}
%
%\end{problem}