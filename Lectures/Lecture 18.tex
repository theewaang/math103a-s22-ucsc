\vspace*{1em}

\begin{mdframed}
\begin{center}
{\Large Sequences \& Series}
\end{center}
\end{mdframed}

\begin{definition}[Sequences]
A \cdef{sequence} of complex numbers is a complex-valued function $z$ whose domain is the set of positive integers. We write $z_n = z(n)$ for the value of the function $z$ at $n$. We think of values occurring in a ``sequential" order
\[z_1,z_2,\ldots,z_n,\ldots\] 
and we usually denote the sequence as $(z_n)_n$.
\end{definition}

\vspace*{1em}

\begin{definition}[Limit of a Sequence]
A sequence $(z_n)_n$ has a limit $L \in \cc$ if for all $\epsilon > 0$, there exists an $N \in \zz_{>0}$ such that 
\[\abs{z - z_n} < \epsilon,\ \text{whenever } n \geq N\]
A sequence that has a limit is called \cdef{convergent} and we write
\[\lim_{n \to \infty} z_n = L \ \text{ or }\ z_n \to L\]
while a sequence with no limit is called \cdef{divergent}.
\[\text{\color{red}add image here}\]
\end{definition}

\vspace*{1em}

\begin{proposition}\label{limuniqueseq}\hfill
\begin{itemize}
\item[(1)] The limit of a convergent sequence is unique.
\item[(2)] For a sequence $(z_n)_n$, we write each term as $z_n = x_n + iy_n$, and extract two real sequences $(x_n)_n$ and $(y_n)_n$. Then, 
\[x_n + iy_n \to x + iy \quad \text{if and only if} \quad x_n \to x\ \text{ and }\ y_n \to y\]
\end{itemize}
\end{proposition}
\begin{proof}
The proof of (1) is similar to that of Theorem \ref{limunique}, and that of (2) is similar to the proof of Theorem \ref{cmplxlimripart}.
%\hfill
%\begin{itemize}
%\item[(1)] Suppose $(z_n)_n$ is a convergent sequence with \[L_1 = \lim_{n \to \infty}\,z_n = L_2\]
%Then, by definition, there exists an $N_1,\,N_2 \in \zz_{>0}$ such that 
%\[\abs{z_n - L_1} < \frac{\epsilon}{2},\ \text{whenever } n \geq N_1;\ \text{ and}\quad \abs{z_n - L_2} < \frac{\epsilon}{2},\ \text{whenever } n \geq N_2\]
%Then, for $n \geq \max\{N_1,N_2\} \geq N_1,N_2$, we have
%\begin{align*}
%\abs{L_1 - L_2} &= \abs{L_1 - z_n + z_n - L_2}\\[0.5em]
%&\leq \abs{z_n - L_1} + \abs{z_n - L_2} \leq \frac{\epsilon}{2} + \frac{\epsilon}{2} = \epsilon
%\end{align*}
%Since $\epsilon > 0$, we get $\abs{L_1 - L_2}$, and therefore $L_1 = L_2$.
%\item[(2)] 
%\end{itemize}
\end{proof}

\vspace*{1em}

\begin{example}
We show that
\[\lim_{n \to \infty}\,-1 + i\,\frac{(-1)^n}{n^2}\]
Proposition \ref{limuniqueseq} tells us that
\begin{align*}
\lim_{n \to \infty}\,-1 + i\,\frac{(-1)^n}{n^2} &= \lim_{n \to \infty}\,-1 + i\,\lim_{n \to \infty}\,\frac{(-1)^n}{n^2}\\[0.5em]
&= -1 + i\,\lim_{n \to \infty}\,\frac{(-1)^n}{n^2}
\end{align*}
So, let's show
\[\lim_{n \to \infty}\,\frac{(-1)^n}{n^2} = 0\]
For any $\epsilon > 0$, choose an $N > 1/\sqrt{\epsilon}$, then for any $n \geq N$ we get
\[\abs{\frac{(-1)^n}{n} - 0} = \abs{\frac{(-1)^n}{n}} = \abs{\frac{1}{n}} = \frac{1}{n^2} \leq \frac{1}{N^2} < \epsilon\]
Thus, 
\[\lim_{n \to \infty}\,-1 + i\,\frac{(-1)^n}{n^2}\]
and hence,
\[\lim_{n \to \infty}\,-1 + i\,\frac{(-1)^n}{n^2}\]
\end{example}

\vspace*{1em}

\begin{definition}[Series]
A \cdef{series} of complex numbers is a \emph{symbol}
\[\sum_{k = 1}^\infty\,z_k = z_1 + z_2 + \cdots + z_n + \cdots\]
associated to the sequence $(z_n)_n$ of complex numbers. A series has an associated \cdef{sequence\ of\ partial} \cdef{sums}
\[s_n = \sum_{k=1}^n\,z_k = \underbrace{\,z_1 + z_2 + \cdots + z_n\,}_{\text{sum of the first $n$ terms}}\]
A series is \cdef{convergent} if $(s_n)_n$ is convergent, this case we write
\[\sum_{k = 1}^\infty\,z_k = \lim_{n \to \infty}\,s_n = \lim_{n \to \infty}\sum_{k=1}^n\,z_k\]
and the limit $\lim_{n \to \infty}\, s_n$ is called the \cdef{sum} of the series. A series that does not converge is said to be \cdef{divergent}.
\end{definition}

\vspace*{1em}

\begin{proposition}\label{seriesrealim}
Suppose that $(z_n)_n$ is a sequence with $z_n = x_n + iy_n$. Then,
\[\sum_{k=1}^\infty \,x_k + iy_k = x + iy \quad \text{if and only if} \quad \sum_{k=1}^\infty \,x_k = x\ \text{ and }\ \sum_{k=1}^\infty \,y_k = y\]
\end{proposition}
\begin{proof}
This is just Proposition \ref{limuniqueseq} (2) applied to the sequences of partial sums.
\end{proof}

\vspace*{1em}

\begin{remark}
According to Proposition \ref{seriesrealim}, we can write
\[\sum_{k=1}^\infty \,x_k + iy_k = \sum_{k=1}^\infty \,x_k + i\sum_{k=1}^\infty \,y_k\]
provided the series on the left \emph{or} the two on the right converge. 
\end{remark}

%\vspace*{1em}

\begin{proposition}[Test for Divergence]\label{tailzero}
If $\sum_{k=1}^\infty \,z_k$ converges, then $z_n \to 0$.
\end{proposition}
\begin{proof}
Write $z_n = x_n + iy_n$, then by Proposition \ref{seriesrealim}, the series $\sum_{k=1}^\infty \,x_k$ and $\sum_{k=1}^\infty \,y_k$ converge. As series of real numbers, we know that $x_n \to 0$ and $y_n \to 0$. Therefore,
\[\lim_{n \to \infty}\,z_n = \lim_{n \to \infty}\,x_n + i\lim_{n \to \infty}\,y_n = 0\]\\[-2em]
\end{proof}

\vspace*{1em}

\begin{corollary}\label{tailbound}
If $\sum_{k=1}^\infty \,z_k$ converges, then there exists an $M > 0$ such that $\abs{z_n} \leq M$ for all $n$. That is, the sequence $(z_n)_n$ is \emph{bounded}.
\end{corollary}
\begin{proof}
If $\sum_{k=1}^\infty \,z_k$ converges, then by Proposition \ref{tailzero}, $z_n \to 0$. Then, we can find an $N$ such that $\abs{z_n} \leq 1$ for every $n \geq N$. Set, 
\[M = \max\set{1,\abs{z_1},\ldots,\abs{z_{N-1}}},\]
then $\abs{z_n} \leq M$, for every $n$. 
\end{proof}

\vspace*{1em}

\begin{definition}
A series $\sum_{k=1}^\infty \,z_k$ is \cdef{absolutely\ convergent} if the series $\sum_{k=1}^\infty \,\abs{z_k}$ of real numbers converges.
\end{definition}

\vspace*{1em}

\begin{corollary}[Absolutely Convergent Series converge]
If $\sum_{k=1}^\infty \,z_k$ is absolutely convergent, then it is convergent.
\end{corollary}
\begin{proof}
By assumption, $\sum_{k=1}^\infty \,\abs{z_k}$ converges. Note that $\abs{x_n} \leq \abs{z_n}$ and $\abs{y_n} \leq \abs{z_n}$ for every $n$. By the comparison test (from calculus), the series
\[\sum_{k=1}^\infty \,\abs{x_k} \quad \text{and} \quad \sum_{k=1}^\infty \,\abs{y_k}\]
converge. Hence, the series $\sum_{k=1}^\infty \,x_k$ and $\sum_{k=1}^\infty \,y_k$ absolutely converge, and thus converge (a result from calculus). By Proposition \ref{seriesrealim}, we conclude that $\sum_{k=1}^\infty \,z_k$ converges.
\end{proof}

\vspace*{1em}

\begin{definition}[Remainder of a Convergent Series]
Suppose $\sum_{k=1}^\infty \,z_k$ is a convergent series and $S$ is its sum. Then {\color{blue}$n^{\text{th}}$} \cdef{remainder} of the series is the complex number
\begin{align*}
\rho_n = S - s_n &= S - \sum_{k=1}^n\,z_k\\[0.5em]
&= \sum_{k=1}^\infty\,z_k - \sum_{k=1}^n\,z_k
\end{align*}
The remainder provides a convenient way to prove that $\sum_{k=1}^\infty\,z_k = S$, as we note that
\[\sum_{k=1}^\infty\,z_k = S \quad \text{if and only if} \quad \rho_n \to 0,\]
as $\abs{S - s_n} = \abs{\rho_n - 0}$.
\end{definition}

%\vspace*{1em}

\begin{mdframed}
\begin{center}
{\Large Power Series}
\end{center}
\end{mdframed}

\begin{definition}[Power Series]
A \cdef{power\ series} is a series
\[\sum_{k=0}^\infty\,a_k(z - z_0)^k = a_0 + a_1(z - z_0) + \cdots + a_n(z - z_0)^n + \cdots\]
where $(a_n)_n$ is a sequence, $z_0 \in \cc$ is fixed, and $z$ is any complex number in a prescribed region in $\cc$. The associated sum, partial sum and remainders depend on $z$, and are denoted $S(z), s_n(z)$ and $\rho_n(z)$ respectively.
\end{definition}

%\vspace*{1em}

\begin{example}
We show that the \emph{geometric series} $\sum_{k=0}^\infty\,az^k$ is convergent when $\abs{z} < 1$. In fact, 
\[\sum_{k=0}^\infty\,a_k(z - z_0)^k = \frac{a}{1 - z}\quad (\abs{z} < 1)\]
We compute the remainder
\begin{align*}
\rho_n(z) &= \frac{a}{1 - z} - s_n(z)\\[1em]
&= \frac{a}{1 - z} - \sum_{k=0}^{n-1}\,az^k\\[1em]
&= \frac{a}{1 - z} - a\left(\frac{1 - z^n}{1 - z}\right)\\[1em]
&= a\left(\frac{z^n}{1 - z}\right)
\end{align*}
Hence, 
\[\abs{\rho_n(z)} = \frac{\abs{a}}{\abs{1 - z}}\cdot \abs{z}^n\]
Note that this sequence of real numbers converges to $0$ if $\abs{z} < 1$ and diverges otherwise. Hence, 
\[\lim_{n \to \infty}\,\rho_n(z) = \begin{cases}0 & \text{if } \abs{z} < 1\\ \text{diverges} & \text{otherwise} \end{cases}\]
\end{example}

\vspace*{1em}

\begin{theorem}[Taylor's Theorem]\label{taylor}
Suppose that $f$ is holomorphic on an open disk $D_R(z_0)$. Then at each $z \in D_R(z_0)$, $f(z)$ has a convergent power series expansion
\[f(z) = \sum_{k=0}^\infty\,a_k(z - z_0)^k\]
with coefficients
\[a_k = \frac{f^{(k)}(z_0)}{k!}\]
The series expansion of $f$ guaranteed by the theorem is called the \cdef{Taylor\ series\ of} {\color{blue}$f$} \cdef{about} {\color{blue}$z_0$}.
\end{theorem}

\vspace*{2em}

\subsection{Problems}
\vspace{0.1in}
To be added
%\begin{problem}\label{prob 12.1}
%
%\end{problem}