\vspace*{1em}

\begin{definition}[Contours]
A \cdef{contour} is an arc consisting of a finite number of smooth arcs joined end to end.\\[0.5em]
A \cdef{simple\ closed\ contour} is a contour that does not cross itself except that the initial and final points are the same.
\end{definition}

\vspace*{1em}

\begin{discussion}[Jordan Curve Theorem]
A deep theorem known as the \emph{Jordan Curve theorem} tells us that every simple closed contour $C$ is the boundary of two distinct domains called the \cdef{interior\ of} {\color{blue}$C$}, which is bounded, and the \cdef{exterior\ of} {\color{blue}$C$}, which is unbounded.
\[\begin{tikzpicture}[scale=0.6]
    \draw[<->,thick] (-2,0)--(5,0);
	\draw[<->,thick] (0,-2)--(0,5);
    \begin{scope}
    \draw[use Hobby shortcut,closed=true,fill=dirt,fill opacity=0.1,draw opacity=0]
	(5.5,5.5) .. (-2.5,5.5) .. (-2.5,-2.5) .. (5.5,-2.5) .. (5.5,5.5);
    \draw[use Hobby shortcut,closed=true,wise arrows,fill=indigo,fill opacity=1/10]
	(4.5,4.5) .. (4,2) .. (3,-2) .. (0,-1) .. (-1.5,-1.5) .. (-1,0) .. (-2,3) .. (-1,4.5) .. (2,4) .. (4.5,4.5);
    \draw[use Hobby shortcut,closed=true,wise arrows,thick]
	(4.5,4.5) .. (4,2) .. (3,-2) .. (0,-1) .. (-1.5,-1.5) .. (-1,0) .. (-2,3) .. (-1,4.5) .. (2,4) .. (4.5,4.5);
    \end{scope}
    \node[label=below:{\large Interior}](A) at (2.25,1.5) {};
    \node[label=below:{\large Exterior}](B) at (0.5,6.5) {};
\end{tikzpicture}\]

The theorem is geometrically evident but the proof is not easy. We will assume its truth so that we can refer to the interior of a simple closed contour.
\end{discussion}

\vspace*{1em}

\begin{mdframed}
\begin{center}
{\Large Contour Integration}
\end{center}
\end{mdframed}

\begin{definition}[Contour Integral]
Suppose $f:G \to \cc$ is a complex function and $C$ is a contour lying in $G$. If $z(t)$, $t \in [a,b]$, is a parametrisation of $C$ and $f(z(t))$ is piecewise continuous, then the \cdef{contour\ integral\ of} {\color{blue}$f$} \cdef{over} {\color{blue}$C$} is
\[\int_C\, f(z)\ dz \coloneqq \int_a^b f(z(t))\,z'(t)\ dt\]
\begin{remark}
Since $C$ is a contour, $z'(t)$ is piecewise continuous and therefore the above integral exists.
\end{remark}
\end{definition}

\vspace*{1em}

\begin{proposition}[Integral is Parametrisation-independent]
Suppose $z:[a,b] \to \cc$ parametrises $C$ and $w:[c,d] \to \cc$ is an orientation-preserving reparametrisation of $C$, then
\[\int_C\,f(z)\ dz = \int_C\, f(w)\ dw\]
\end{proposition}
\begin{proof}
By definition of an orientation-preserving reparametrisation, there exists a surjective map $\phi:[c,d] \to [a,b]$ such that $\phi(c) = a,\,\phi(d) = b,\, \phi'(s) > 0$ and $w(s) = \phi(z(s))$. Then
\begin{align*}
\int_C\, f(w)\ dw &= \int_c^d f(w(s))\,w'(s)\ ds\\[0.5em]
 &= \int_c^d f(z(\phi(s)))\,\phi'(z(s))\,z'(s)\ ds,\ \text{apply chain rule to $w(s) = \phi(z(s))$}\\[0.5em]
 &= \int_a^b f(z(t))\,z'(t)\ dt,\ \text{set $t = \phi(s)$}\\[0.5em]
 &= \int_C\, f(z)\ dz
\end{align*}
\end{proof}

\vspace*{1em}

\begin{discussion}[Notation for Contours]\hfill
\begin{itemize}
\item[(1)] Suppose $C$ is a contour, then $-C$ denotes the same set of points as $C$ but with opposite orientation. If $z:[a,b] \to \cc$ is a parametrisation of $C$, then $w:[-b,-a] \to \cc$ defined as $w(t) \coloneqq z(-t)$ is a parametrisation of $-C$.
\[\begin{tikzpicture}[scale=0.55]
    \draw[<->,thick] (-2,0)--(5,0);
	\draw[<->,thick] (0,-2)--(0,5);
    \begin{scope}
        \node[label=left:{\color{teal}$C$}](A) at (-2,-2) {};
        \node[label=right:{\color{firebrick}$-C$}](B) at (4.5,4.5) {};
        \draw[use Hobby shortcut,counterclockwise arrows,clockwise arrows,thick]
	(4.5,4.5) .. (3,2) .. (0,1) .. (-2,-2);
    \end{scope}
    \draw [fill=black] (A) circle (2pt);
    \draw [fill=black] (B) circle (2pt);
\end{tikzpicture}\]
\item[(2)] If $C_1$ is a contour from $z_1$ to $z_2$ and $C_2$ is a contour from $z_2$ to $z_3$, then their \cdef{sum} $C = C_1 + C_2$ is the contour obtained by transversing $C_1$ and then $C_2$.

\begin{center}
\begin{minipage}{0.4\textwidth}
\[\begin{tikzpicture}[scale=0.6]
    \draw[<->,thick] (-2,0)--(5,0);
	\draw[<->,thick] (0,-2)--(0,5);
    \begin{scope}
        \node[label=left:{\color{teal}$C_1$}](A) at (-3,-2) {};
        \node[label={[xshift=3.5em,yshift=-1.5em]\color{indigo}$C = C_1 + C_2$}](B) at (1,1){};
        \node[label=right:{\color{firebrick}$C_2$}](C) at (4.5,4.5){};
        \draw[use Hobby shortcut,clockwise arrows, clockwise arrowsmore,thick]
	(1,1) .. (0.5,0.5) .. (-1.5,1) .. (-3,-2);
        \draw[use Hobby shortcut,clockwise arrowsnew, clockwise arrowsmore,thick]
	(4.5,4.5) .. (2.5,2.5) .. (3,1.5) .. (2.5,2.5) .. (1,1);
    \end{scope}
    \draw [fill=black] (A) circle (2pt);
    \draw [fill=black] (B) circle (2pt);
    \draw [fill=black] (C) circle (2pt);
\end{tikzpicture}\]
\end{minipage} \hspace*{2em} 
\begin{minipage}{0.4\textwidth}
\[\begin{tikzpicture}[scale=0.6]
    \draw[<->,thick] (-2,0)--(5,0);
	\draw[<->,thick] (0,-2)--(0,5);
    \begin{scope}
        \node[label=left:{\color{teal}$C_1$}](A) at (-2,-1.5) {};
        \node[label=right:{\color{indigo}$C = C_1 - C_2$}](B) at (2.5,2.5){};
        \node[label=right:{\color{firebrick}$C_2$}](C) at (5,-0.5){};
        \draw[use Hobby shortcut,clockwise arrows, clockwise arrowsmore,thick]
	(2.5,2.5) .. (-0.5,1.5) .. (-1,0) .. (-2,-1.5);
        \draw[use Hobby shortcut,counterclockwise arrows2, clockwise arrowsmorenew,thick]
	(5,-0.5) .. (4,-1) .. (3.5,-1.5) .. (1.5,-1.5) .. (1,-0.5) .. (1,1) .. (2.5,2.5);
    \end{scope}
    \draw [fill=black] (A) circle (2pt);
    \draw [fill=black] (B) circle (2pt);
    \draw [fill=black] (C) circle (2pt);
\end{tikzpicture}\]
\end{minipage}
\end{center}

If $C_1$ and $C_2$ have the same final point, then we can consider the sum of $C_1$ and $-C_2$ and is written as $C_1 - C_2 \coloneqq C_1 + (-C_2)$.
\end{itemize}
\end{discussion}

%\vspace*{1em}

\begin{proposition}[Properties of Contour Integral]\label{contprop}
Assume $f,\,g$ are piecewise continuous on the contours we consider below.
\begin{itemize}[itemsep=0.5em]
\item[(1)] $\displaystyle \int_C\,z_0 f(z)\ dz = z_0\int_C\, f(z)\ dz$, for any $z_0 \in \cc$.
\item[(2)] $\displaystyle \int_C\, f(z) + g(z)\ dz = \int_C\, f(z)\ dz + \int_C g(z)\ dz$.
\item[(3)] $\displaystyle \int_{-C}\, f(z)\ dz = -\int_C\,f(z)\ dz$.
\item[(4)] $\displaystyle \int_C\, f(z)\ dz = \int_{C_1}\, f(z)\ dz + \int_{C_2}\, f(z)\ dz$ if $C = C_1 + C_2$.
\end{itemize}
\end{proposition}
\begin{proof}\hfill
\begin{itemize}
\item[(1)] Suppose $C$ is parametrised by $z:[a,b] \to \cc$
\begin{align*}
\int_C\,z_0 f(z)\ dz &=  \int_C\,z_0 f(z(t))\,z'(t)\ dz\\[0.5em]
&=  z_0\int_a^b\, f(z(t))\,z'(t)\ dz,\ \text{by Proposition \ref{paraint} (1)}\\[0.5em]
&=  z_0\int_C\, f(z)\ dz
\end{align*}
\item[(2)] This will follow from Proposition \ref{paraint} (2).
\item[(3)] Suppose $C$ is parametrised by $z:[a,b] \to \cc$, then, as we note before, a parametrisation of $-C$ is $w:[-b,-a] \to \cc$ where $w(t) = z(-t)$. Then
\begin{align*}
\int_{-C}\,f(w)\ dw &= \int_{-b}^{-a}\, f(w(t))\,w'(t)\ dt\\[0.5em]
 &= -\int_{-b}^{-a}\, f(z(-t))\,z'(-t)\ dt,\ \text{apply chain rule to $w(t) = z(-t)$}\displaybreak\\[0.5em]
 &= -\int_{-a}^{-b}\, f(z(-t))\,z'(-t)\ dt,\ \text{by Proposition \ref{paraint} (4)}\\[0.5em]
 &= \int_{a}^{b}\, f(z(s))\,z'(s)\ ds,\ \text{set $s=-t$}\\[0.5em]
 &= \int_C\, f(z)\ dz
\end{align*}
\item[(4)] We leave this as an exercise (Problem \ref{prob 12.1}) for the motivated student.
\end{itemize}
\vspace*{-\baselineskip}
\end{proof}

\vspace*{1em}

\begin{example}\hfill
\begin{itemize}[itemsep=1.5em]
\item[(1)] Integrate $f(z) = \dfrac{1}{z}$ over the following contours:
\begin{itemize}
\item[$\bullet$] $C_1$: upper semicircle of the unit circle, from $1$ to $-1$.
\item[$\bullet$] $C_2$: lower semicircle of the unit circle, from $1$ to $-1$.
\item[$\bullet$] $C_3$: $C_1 - C_2$.
\end{itemize}
\begin{minipage}{0.6\textwidth}
For $C_1$, parametrise $C_1$ as $z(t) = e^{it},\ 0 \leq t \leq \pi$. Then
\[\int_{C_1}\,\frac{1}{z}\ dz = \int_0^\pi \frac{1}{e^{it}}\, ie^{it}\ dt = i\int_0^\pi\,dt = \pi i\]
\end{minipage}
\begin{minipage}{0.3\textwidth}
\[\begin{tikzpicture}[scale=1.3]
    \draw[<->,thick] (-1.75,0)--(1.75,0);
	\draw[<->,thick] (0,-0.5)--(0,1.75);
    \begin{scope}
        \node[label=right:{$C_1$}](A) at (1,1) {};
        \draw[use Hobby shortcut,clockwise arrows,thick]
	(-1.25,0) .. (0,1.25) .. (1.25,0);
    \end{scope}
\end{tikzpicture}\]
\end{minipage}

\vspace*{1em}

\begin{minipage}{0.6\textwidth}
For $C_1$, parametrise $C_1$ as $z(t) = e^{-it},\ 0 \leq t \leq \pi$. Then
\[\int_{C_1}\,\frac{1}{z}\ dz = \int_0^\pi \frac{1}{e^{-it}}\, (-ie^{-it})\ dt = -i\int_0^\pi\,dt = -\pi i\]
\end{minipage}
\begin{minipage}{0.3\textwidth}
\[\begin{tikzpicture}[scale=1.3]
    \draw[<->,thick] (-1.75,0)--(1.75,0);
	\draw[<->,thick] (0,-1.75)--(0,0.5);
    \begin{scope}
        \node[label=right:{$C_2$}](A) at (1,-1) {};
        \draw[use Hobby shortcut,clockwise arrows,thick]
	(-1.25,0) .. (0,-1.25) .. (1.25,0);
    \end{scope}
\end{tikzpicture}\]
\end{minipage}

\vspace*{1em}

\begin{minipage}{0.6\textwidth}
For $C_1$, parametrise $C_1$ as $z(t) = e^{-it},\ 0 \leq t \leq \pi$. Then
\begin{align*}
\int_{C_3}\,\frac{1}{z}\ dz = \int_{C_1 - C_2}\,\frac{1}{z}\ dz &= \int_{C_1}\,\frac{1}{z}\ dz + \int_{- C_2}\,\frac{1}{z}\ dz\\[0.5em]
 &= \int_{C_1}\,\frac{1}{z}\ dz - \int_{C_2}\,\frac{1}{z}\ dz\\[0.5em]
 &= \pi i - (-\pi i)\\[0.5em]
 &= 2\pi i
\end{align*}
\end{minipage}
\begin{minipage}{0.3\textwidth}
\[\begin{tikzpicture}[scale=1.3]
    \draw[<->,thick] (-1.75,0)--(1.75,0);
	\draw[<->,thick] (0,-1.75)--(0,1.75);
    \begin{scope}
        \node[label=right:{$C_3$}](A) at (1,1) {};
        \draw[use Hobby shortcut,clockwise arrows,thick]
	(-1.25,0) .. (0,1.25) .. (1.25,0) .. (0,-1.25) .. (-1.25,0);
    \end{scope}
\end{tikzpicture}\]
\end{minipage}

\emph{This example shows that the integral may depend on the path taken and not just on the endpoints. Also, the integral over a closed contour may be non-zero.}

\item[(2)] Integrate $f(z) = z$ over \emph{any} contour $C$ connecting a point $w_1$ to a point $w_2$.\\[0.5em]
First, suppose $C$ is a smooth arc joining $w_1$ and $w_2$ with parametrisation $z:[a,b] \to \cc$.
\[\begin{tikzpicture}[scale=0.55]
    \draw[<->,thick] (-2,0)--(5,0);
	\draw[<->,thick] (0,-2)--(0,5);
    \begin{scope}
        \node[label=below:{$w_1$}](A) at (-2,-2) {};
        \node[label=left:{$w_2$}](B) at (2.5,3.5) {};
        \draw[use Hobby shortcut,clockwise arrows,thick]
	(2.5,3.5) .. (4,2) .. (0,2) .. (-1,1) .. (0.5,-0.5) .. (-1.5,-1) .. (-2,-2);
    \end{scope}
    \draw [fill=black] (A) circle (2pt);
    \draw [fill=black] (B) circle (2pt);
\end{tikzpicture}\]
Since,\[\left(\frac{z(t)^2}{2}\right)' = \frac{z'(t)z(t) + z(t)z'(t)}{2} = z(t)z'(t).\]
Therefore,
\begin{align*}
\int_C\,f(z)\ dz = \int_C\,z\ dz &= \int_a^b\,z(t)\,z'(t)\ dt\\[0.5em]
 &= \frac{z(b)^2}{2} - \frac{z(a)^2}{2},\ \text{by Proposition \ref{pathftc}}\\[0.5em]
 &= \frac{w_2^2 - w_1^2}{2}
\end{align*}

%\begin{minipage}{0.6\textwidth}
%Since,\[\left(\frac{z(t)^2}{2}\right)' = \frac{z'(t)z(t) + z(t)z'(t)}{2} = z(t)z'(t).\]
%Therefore,
%\begin{align*}
%\int_C\,f(z)\ dz = \int_C\,z\ dz &= \int_a^b\,z(t)\,z'(t)\ dt\\[0.5em]
% &= \frac{z(b)^2}{2} - \frac{z(a)^2}{2},\ \text{by Proposition \ref{pathftc}}\\[0.5em]
% &= \frac{w_2^2 - w_1^2}{2}
%\end{align*}
%\end{minipage}\hspace*{1em}
%\begin{minipage}{0.3\textwidth}
%\[\begin{tikzpicture}[scale=0.5]
%    \draw[<->,thick] (-2,0)--(5,0);
%	\draw[<->,thick] (0,-2)--(0,5);
%    \begin{scope}
%        \node[label=below:{$w_1$}](A) at (-2,-2) {};
%        \node[label=left:{$w_2$}](B) at (2.5,3.5) {};
%        \draw[use Hobby shortcut,clockwise arrows,thick]
%	(2.5,3.5) .. (4,2) .. (0,2) .. (-1,1) .. (0.5,-0.5) .. (-1.5,-1) .. (-2,-2);
%    \end{scope}
%    \draw [fill=black] (A) circle (2pt);
%    \draw [fill=black] (B) circle (2pt);
%\end{tikzpicture}\]
%\end{minipage}

\vspace*{1em}

Now, if $C$ is a contour, we can write $C = C_1 + \cdots + C_n$, where $C_i$ is a smooth arc joining $z_i$ to $z_{i+ 1}$ with $z_1 = w_1$ and $z_{n+1} = w_2$. Then,
\begin{align*}
\int_C\,z\ dz &= \sum_{i=1}^n\int_{C_i}\,z\ dz,\ \text{by Proposition \ref{contprop} (4)}\\[0.5em]
 &= \sum_{i=1}^n \frac{z_{i+1}^2 - z_i^2}{2}\\[0.5em]
 &= \frac{z_{n+1}^2 - z_1^2}{2}\\[0.5em]
 &= \frac{w_2^2 - w_1^2}{2}
\end{align*}

\emph{This example shows that some integrals do depend only on the end points and not the path taken. Also, for any contour $C$ is closed, that is, when $w_2 = w_1$, we have shown hence that
\[\int_C z\ dz = 0.\]}

\item[(3)] Integrate $f(z) = z^m\overline{z}^n$, for $m,n \in \zz$, over the unit circle $C$.
\[\begin{tikzpicture}[scale=1.4]
    \draw[<->,thick] (-1.75,0)--(1.75,0);
	\draw[<->,thick] (0,-1.75)--(0,1.75);
    \begin{scope}
        \node[label=right:{$C$}](A) at (1,1) {};
        \draw[use Hobby shortcut,clockwise arrows,thick]
	(-1.25,0) .. (0,1.25) .. (1.25,0) .. (0,-1.25) .. (-1.25,0);
    \end{scope}
\end{tikzpicture}\]
Parametrise $C$ as $z(t) = e^{it},\ 0 \leq t \leq 2\pi$. Then,\newpage
\begin{align*}
\int_C\,f(z)\ dz &= \int_C\,z^m\overline{z}^n\ dz\\[0.5em]
 &= \int_0^{2\pi}\,(e^{it})^m(\overline{e^{it}})^n\,ie^{it}\ dt\\[0.5em]
 &= i\int_0^{2\pi}\,(e^{it})^m(e^{-it})^n\,e^{it}\ dt\\[0.5em]
 &= i\int_0^{2\pi}\,e^{imt}e^{-int}e^{it}\ dt\\[0.5em]
 &= i\int_0^{2\pi}\,e^{(m - n + 1)it}\ dt
\end{align*}

\vspace*{2em}

\begin{minipage}[t]{0.45\textwidth}
Case I.\ $m = n-1$
\begin{align*}
\int_C\,f(z)\ dz &= i\int_0^{2\pi}\,e^{(m - n + 1)it}\\[0.5em]
 &= i\int_0^{2\pi}\ dt\\[0.5em]
 &= 2\pi i
\end{align*}
\end{minipage}\hspace*{2em}
\begin{minipage}[t]{0.4\textwidth}
Case II.\ $m \neq n-1$
\begin{align*}
\int_C\,f(z)\ dz &= i\int_0^{2\pi}\,e^{(m - n + 1)it}\\[0.5em]
 &= i\Bigg[\frac{e^{(m - n + 1)it}}{i(m - n + 1)}\Bigg]_0^{2\pi}\\[0.5em]
 &= \frac{1}{m - n + 1}\left(e^{2(m - n + 1)\pi i} - e^0\right)\\[0.5em]
 &= \frac{1}{m - n + 1}\,(1 - 1)\\[0.5em]
 &= 0
\end{align*}
\end{minipage}

%\begin{minipage}{0.6\textwidth}
%Parametrise $C$ as $z(t) = e^{it},\ 0 \leq t \leq 2\pi$. Then,
%\begin{align*}
%\int_C\,f(z)\ dz = \int_C\,z^m\overline{z}^n\ dz &= \int_0^{2\pi}\,(e^{it})^m(\overline{e^{it}})^n\,ie^{it}\ dt\\[0.5em]
% &= i\int_0^{2\pi}\,(e^{it})^m(e^{-it})^n\,e^{it}\ dt\\[0.5em]
% &= i\int_0^{2\pi}\,e^{imt}e^{-int}e^{it}\ dt\\[0.5em]
% &= i\int_0^{2\pi}\,e^{(m - n + 1)it}\ dt\\[1em]
%\text{Case I.\ $m = n-1$}\\[0.5em]
% &= i\int_0^{2\pi}\ dt\\[0.5em]
% &= 2\pi i\\[1em]
%\text{Case II.\ $m \neq n-1$}\\[0.5em]
% &= i\Bigg[\frac{e^{(m - n + 1)it}}{i(m - n + 1)}\Bigg]_0^{2\pi}\\[0.5em]
% &= \frac{1}{m - n + 1}\left(e^{2(m - n + 1)\pi i} - e^0\right)\\[0.5em]
% &= \frac{1}{m - n + 1}\,(1 - 1)\\[0.5em]
% &= 0
%\end{align*}
%\end{minipage}\hspace*{1em}
%\begin{minipage}{0.3\textwidth}
%
%\end{minipage}
\end{itemize}
\end{example}

\vspace*{2em}

\subsection{Problems}
\vspace{0.1in}
To be added
%\begin{problem}\label{prob 12.1}
%
%\end{problem}